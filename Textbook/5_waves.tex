\chapter{Waves}

%
%  SECTION 5.1 - Gravity and Capillary Waves
%

\section{Gravity and Capillary Waves}


% 5.1.1 - Surface Waves (examples of waves, generation of surface waves, gravity and capillary waves and both combined, drawing of a wave and definitions)

\subsection{Surface Waves}

We turn our attention now to the \emph{interface} between two fluids -- in our discussion here, almost always this will be water and air.  In that case the interface is called a \emph{surface wave}.  Surface waves can be generated by wind, ocean tides, surface disturbances, and underwater disturbances (an earthquake, for example, giving rise to a tsunami).  Figure \ref{fig_pond} shows an example of wind-generated waves on a pond.  

\begin{figure}
\centering\includegraphics[width=0.9\linewidth]{Figures/Chapter5/fig_pond_waves}
\caption{Small amplitude waves generated by wind on the surface of a small pond on the cmapus of Ontario Tech University.}
\label{fig_pond}
\end{figure}

These waves are called \emph{gravity waves} since, as we'll see, the main force determining their behaviour is gravity.  However, in some cases -- typically when the wavelengths are very short -- \emph{surface tension} can be the dominant force in attempting to restore equilibrium to the surface.  In that case the waves can behave very differently, and often both forces are important in determining the dynamics of waves.

To begin our discussion of waves, consider a generic example of a surface wave as shown in Figure \ref{fig_generic_wave}.  The shape of the surface -- the interface between the two fluids -- will be described by the function $\eta(x, t)$, so that the surface itself is given by the equation
\begin{equation}
y = \eta(x, t).
\end{equation}
Note that we're setting the direction of gravity to be downward along the negative $y$ direction, in which case $\vec{g} = [0, -g, 0]$ in this coordinate system.

\begin{figure}
\centering\includegraphics[width=0.8\linewidth]{Figures/Chapter5/fig_generic_wave}
\caption{The surface of a wave is described by the function $\eta(x, t)$.}
\label{fig_generic_wave}
\end{figure}

We'll assume that the fluid is two-dimensional, so there will be no $z$ dependence or flow in that direction:
\[
\vec{u} = [u(x, y, t), v(x, y, t), 0].
\]
We'll also take the flow to be irrotational, so that $\curl \vec{u} = \vec{0}$.  In two dimensions, this means
\[
\dfdx{v}{x} - \dfdx{u}{y} = 0.
\]
It might seem like a leap to assume irrotationality, but if we imagine starting with a perfectly still fluid, with no wave at the interface, then the flow is obviously initially irrotational.  But as time goes on, Kelvin's circulation theorem (see Section \ref{sec_circulation}) guarantees that the flow will remain irrotational -- the circulation must remain zero.

Since our flow is irrotational, we're free to describe it with the velocity potential $\varphi(x, y, t)$.  In two dimensions, $\vec{u} = \grad \varphi$ becomes
\begin{equation}
u = \dfdx{\varphi}{x} \quad \text{and} \quad \dfdx{\varphi}{y}.
\end{equation}
Finally, we're still dealing with incompressible fluids, which means the velocity potential must satisfy Laplace's equation:
\begin{equation}
\ddfdx{\varphi}{x} + \ddfdx{\varphi}{y} = 0.
\end{equation}
This lays out the guiding equations behind our discussion of waves; all that's left is to determine the boundary conditions.

% 5.1.2 - The Kinematic and Pressure Conditions

\subsection{The Kinematic and Pressure Conditions}

The first boundary condition comes from the fact that any fluid particle at the surface must remain at the surface.  You might be able to visualize this by imagining dyeing the fluid at the surface; as the surface moves up or down, the dyed fluid stays at the surface.  This means that those fluid particles at the surface must follow the vertical motion of the wave.  We can model this mathematically by defining a new function, $F(x, y, t)$, such that
\begin{equation}
F(x, y, t) = y - \eta(x, t).
\end{equation}
Although in general $F$ can take on any value, all fluid particles at the surface have $F =$ constant -- namely, $F(x, y, t) = 0$ there since $y = \eta(x, t)$ defines our surface.

With the function $F$ constant at the surface, we can obviously write
\[
\frac{DF}{Dt} = 0 \quad \text{on} \quad y = \eta(x, t)
\]
or, expanding the material derivative,
\begin{equation}
\label{eq_kin_cond_md}
\dfdx{F}{t} + (\vec{u} \cdot \grad ) F = 0 \quad \text{on} \quad y = \eta(x, t).
\end{equation}
Now,
\[
\dfdx{F}{t} = \frac{\partial}{\partial t} \Bigl( y - \eta(x, t) \Bigr) = -\dfdx{\eta}{t}
\]
and
\[
(\vec{u} \cdot \grad) F = u \dfdx{F}{x} + v\dfdx{F}{y} = -u \dfdx{\eta}{x} + v,
\]
so equation (\ref{eq_kin_cond_md}) becomes
\[
-\dfdx{\eta}{t} - u\dfdx{\eta}{x} + v = 0,
\]
or
\begin{equation}
\label{eq_kin_cond}
\boxed{
\dfdx{\eta}{t} + u\dfdx{\eta}{x} = v \quad \text{on} \quad y = \eta(x, t).
}
\end{equation}
This is called the \emph{kinematic condition} at the surface; it ensures that fluid particles at the surface move vertically as the surface does.

The second boundary condition at the surface follows from considering the pressure in the fluid:  it must be \emph{continuous} across the interface.  Since we're usually assuming air above the fluid, we'll take the pressure at the surface to be atmospheric pressure, $p_0$, which is of course constant along the surface.  

Consider Euler's equation in the form of equation \ref{eq_euler_bernoulli},
\[
\frac{\partial \uu}{\partial t} + (\curl \uu ) \times \uu = -\grad \left( \frac{p}{\rho} + \tfrac{1}{2} \uu^2 + \Phi \right).
\]
The second term on the left is zero here, since we have irrotational flow, and we can write $\vec{u} = \grad \varphi$ and then combine the velocity potential with the other terms in the gradient on the right.  We then have
\[
\grad \left( \dfdx{\varphi}{t} + \frac{p}{\rho} + \frac{1}{2} \vec{u}^2 + \Phi \right) = 0.
\]
Integrating leads to
\begin{equation}
\label{eq_euler_wave}
\dfdx{\varphi}{t} + \frac{p}{\rho} + \frac{1}{2} \vec{u}^2 + \Phi = G(t),
\end{equation}
where $G(t)$ is the integration ``constant'' -- it can still be a function of time.  But we're free to take $G(t)$ to be anything we want, since it will go away once we take the space derivative. So, for reasons we'll see in a second, let's take it to be
\begin{equation}
G(t) = \frac{p_0}{\rho}.
\end{equation}
Note that that's the constant atmospheric pressure $p_0$ in that equation.

Our next step is to evaluate equation (\ref{eq_euler_wave}) at the surface $y = \eta(x, t)$:
\[
\dfdx{\varphi}{t} + \frac{p_0}{\rho} + \frac{1}{2} \vec{u}^2 + \Phi = \frac{p_0}{\rho} \quad \text{on} \quad y = \eta(x, t).
\]
Now we see why that choice of $G(t)$ was made, since the pressure terms cancel out.  Writing $\Phi = gy$ and $\vec{u}^2 = u^2 + v^2$, this becomes
\begin{equation}
\label{eq_pressure_cond}
\boxed{
\dfdx{\varphi}{t} +  \frac{1}{2}(u^2 + v^2) + g\eta = 0 \quad \text{on} \quad y = \eta(x, t).
}
\end{equation}
This is the \emph{pressure condition}, the second boundary condition that must be satisfied by the wave surface.  Together with the kinematic condition and Laplace's equation, they make up the theory of surface waves.  Unfortunately, although Laplace's equation is linear, the boundary conditions are not, and this makes them rather difficult to work with.  


% 5.1.3 - Small Amplitude Waves (linearization)

\subsection{Small Amplitude Waves}

If we restrict our investigation to waves that have an amplitude that is small compared to the wavelength (see Figure \ref{fig_small_wave}), we can make some simplifications.  Let's suppose that the fluid velocities $u$ and $v$ are small, and that therefore the surface $\eta$ doesn't deviate too much from $y = 0$.  If that's the case, then we can assume the derivative $\partial \eta / \partial x$ is \emph{also} small, since the slope of the wave won't deviate too much from horizontal.  If any of the terms in equations (\ref{eq_kin_cond}) or (\ref{eq_pressure_cond}) have two or more of these small quantities multiplied together, we'll drop those terms -- in other words, we'll drop anything that is quadratic or higher in smallness.  This is called \emph{linearizing} the boundary conditions.

\begin{figure}
\centering\includegraphics[width=0.8\linewidth]{Figures/Chapter5/fig_small_wave}
\caption{A small-amplitude wave has a wavelength $\lambda \gg A$.}
\label{fig_small_wave}
\end{figure}

Let's start with the kinematic condition, equation (\ref{eq_kin_cond}).  The second term can be dropped since both $u$ and $\partial \eta / \partial x$ are small.  But we can do one other thing, as well.  Let me  write out the condition, being sure to use $y = \eta(x, t)$ in place of $y$:
\[
v(x, \eta, t) = \dfdx{\eta}{t}.
\]
Since $\eta$ is close to zero, we can expand the velocity $v$ about $y = \eta = 0$ in a Taylor expansion, and get
\[
v(x, \eta, t) \approx v(x, 0, t) + \eta \dfdx{v}{y} \bigg|_{\eta=0},
\]
where the higher terms in the expansion were dropped.  In fact, we can drop the second term, as well, since both $\eta$ and $v$ are small, and the kinematic condition becomes
\[
v(x, 0, t) = \dfdx{\eta}{t} \quad \text{on} \quad y=0.
\]
One last thing: we can write this in terms of the velocity potential rather than the velocity using $v = \partial \varphi / \partial y$ and get
\begin{equation}
\label{eq_kin_cond_lin}
\boxed{
\dfdx{\varphi}{y} = \dfdx{\eta}{t} \quad \text{on} \quad y=0.
}
\end{equation}

We can make similar arguments for the pressure condition, equation (\ref{eq_pressure_cond}).  The linearized version is
\begin{equation}
\label{eq_pressure_cond_lin}
\boxed{
\dfdx{\varphi}{t} + g\eta = \quad \text{on} \quad y=0.
}
\end{equation}



% 5.1.4 - Sinusoidal Waves (dispersion relation, potential, example of fluid path under wave)

\subsection{Sinusoidal Waves}

As a simple but important example, suppose the surface can be described by a sinusoidal travelling wave, given by
\begin{equation}
\eta(x, t) = A \cos(kx - \omega t),
\end{equation}
where $k = 2\pi / \lambda$ is the wave number and $\omega = \nu / 2\pi$ is the angular frequency of the wave.

The boundary conditions will not only allow us to find the velocity potential of the fluid, but also will place important restrictions on the behaviour of the waves.  The linearized kinematic condition says
\[
\dfdx{\varphi}{y} = \dfdx{\eta}{t} = \omega A \sin(kx - \omega t),
\]
suggesting that the potential takes the form
\begin{equation}
\label{eq_pot_suggestion}
\varphi(x, y, t) \propto f(y) \sin (kx - \omega t).
\end{equation}
We can find the unknown function $f(y)$ by ensuring that the potential satisfies Laplace's equation,
\[
\ddfdx{\varphi}{x} + \ddfdx{\varphi}{y} = 0.
\]
Substituting equation (\ref{eq_pot_suggestion}) into Laplace's equation leads to 
\[
-k^2 f(y) \sin (kx - \omega t) + f''(y) \sin (kx - \omega t) = 0.
\]
Cancelling out the sine term and rearranging gives the familiar differential equation
\[
f'' = k^2 f,
\]
with general solution
\[
f(y) = C e^{ky} + D e^{-ky}.
\]

Although we've discussed at length the boundary conditions at the surface of the fluid, we haven't mentioned what happens as we go down in depth.  Two options are possible:  that the fluid extends to infinity below the surface (this is called \emph{deep water waves}) or that a ``floor'' exists at some depth $y = -h$.  We'll use the first possibility here, and you can explore the effects of finite depth in Problem \ref{prob_finite_depth}.

If the fluid extends down to $y \to -\infty$, then we have to drop the second term of our solution by setting $D = 0$ -- otherwise the potential will blow up as we go deeper into the fluid.  We have, therefore, 
\[
\varphi(x, y, t) = C e^{ky} \sin (kx - \omega t).
\]
We can find the constant $C$ by using the kinematic pressure condition, equation (\ref{eq_kin_cond_lin}); taking the derivatives and setting $y = 0$ gives
\[
kC \sin(kx - \omega t) = \omega A \sin (kx - \omega t),
\]
or
\begin{equation}
kC = \omega A.
\end{equation}
Thus, the velocity potential is
\begin{equation}
\varphi(x, y, t) = \frac{\omega A}{k} e^{ky} \sin (kx - \omega t).
\end{equation}
This holds \emph{everywhere} in the fluid, not just at the surface, and we can use it to explore the motion of the fluid below the waves -- we'll do that in a moment.  But first, what about the pressure condition?  We haven't applied that yet.

Evaluating the pressure condition, equation (\ref{eq_pressure_cond_lin}), gives
\[
-\frac{\omega^2}{k} \cos(kx - \omega t) + gA \cos(kx - \omega t) = 0,
\]
or
\begin{equation}
\label{eq_disp_relation}
\boxed{
\omega^2 = gk.
}
\end{equation}
This is an important relationship between the frequency of a wave and its wavelength, and is called the \emph{dispersion relation} for deep water gravity waves.

Now, for waves on a string, the dispersion relation is
\[
\omega = vk,
\]
and for light waves it's $\omega = ck$.  Those waves are in fact \emph{dispersionless}, with $\omega \propto k$.  In that case the speed of the wave is the same regardless of wavelength, but for water waves, with $\omega \propto \sqrt{k}$, different wavelengths will travel at different speeds.  For a sinusoidal travelling wave, the wave speed is given by
\begin{equation}
c = \frac{\omega}{k},
\end{equation}
meaning that for water waves
\[
c = \sqrt{\frac{g}{k}} = \sqrt{\frac{g\lambda}{2\pi}}.
\]
Thus, longer wavelengths will travel \emph{faster} than shorter ones.  This leads to some interesting surface wave effects, one of which is shown in Figure \ref{fig_gravity_ripple} -- the waves with longer wavelengths will travel further from the source of the disturbance in the pond, giving the ripples a distinctive look.

\begin{example}[Fluid motion under the surface]

Now that we have the velocity potential $\varphi(x, y, t)$, we can examine the entire fluid flow.  In particular, imagine a single fluid particle initially at some depth under the surface.  What path does the particle take as the surface waves propagate to the right?

Start with the fluid velocity, which we find from the potential:
\[
u = \dfdx{\varphi}{x} = A \omega e^{ky} \cos (kx - \omega t),
\]
and
\[
v = \dfdx{\varphi}{y} = A \omega e^{ky} \sin (kx - \omega t).
\]
Fluid particles, of course, follow the fluid, so we can write the change in position of a particular fluid particle as
\[
\frac{dx}{dt} = u \quad \text{and} \quad \frac{dy}{dt} = y.
\]
This gives us two coupled differential equations to solve.  But we can uncouple them by making a reasonable assumption:  that the fluid particle doesn't deviate too much from its \emph{mean position} $(\bar{x}, \bar{y})$.  That is, 
\[
x = \bar{x} + x'  \quad \text{and} \quad y = \bar{y} + y',
\]
where $x'$ and $y'$ are small.

Since $\bar{x}$ and $\bar{y}$ are constant, we can write
\[
\frac{dx}{dt} = \frac{dx'}{dt} = u \approx A \omega e^{k\bar{y}} \cos (k \bar{x} - \omega t)
\]
and 
\[
\frac{dy}{dt} = \frac{dy'}{dt} = v \approx A \omega e^{k\bar{y}} \sin (k \bar{x} - \omega t).
\]
Note that now the right-hand-sides contain only the (constant) mean positions, meaning that we can simply integrate to find the particle path and get
\begin{equation}
x' = -A e^{k\bar{y}} \sin(k\bar{x} - \omega t), \quad  y' = A e^{k\bar{y}} \cos (k\bar{x} - \omega t).
\end{equation}
As time goes on and the surface wave travels to the right, the fluid particles under the surface trace out \emph{circles} of radius $Ae^{k\bar{y}}$.  As the depth increases, the circles get smaller, as shown in Figure \ref{fig_circle_paths}.

\end{example}


\begin{figure}
\centering\includegraphics[width=0.8\linewidth]{Figures/Chapter5/fig_circle_paths}
\caption{Fluid particles in deep water under surface waves move in circular paths with radii that get exponentially smaller with decreasing depth.  If the wave is propagating to the right, the paths will be clockwise.}
\label{fig_circle_paths}
\end{figure}

% 5.1.5 - Capillary Waves (surface tension effects, new BCs, surface tension-dominated waves, etc; include pic of the two types of waves, plus downstream/upstream stuff and pic of that)


%
%  SECTION 5.2 - Group Waves
%

% 5.2.1 - Wave Packets

% 5.2.2 - Group Velocity


%
%  SECTION 5.3 - Sound waves
%

% 5.3.1 - Compressible Fluids

% 5.3.2 - Small Amplitude Sound waves





\section*{Problems}
\addcontentsline{toc}{section}{Problems}
\markright{Problems}%

\begin{problem}[Effects of finite depth]
\label{prob_finite_depth}
To come
\end{problem}

