\chapter{Special Topics}

%
%  SECTION 6.1 - Navier-Stokes
%

\section{Deriving the Navier-Stokes Equation}
\label{sec_ns_deriv}

As we'll see, the Navier-Stokes equation follows directly from Newton's second law; however, because it depends on the \emph{deformation} of fluid elements, we need a whole new framework to describe the forces at work. 

\subsection{The Stress Tensor and Vector}
\label{sec_stress}

To start, let's consider a fluid element -- I'll draw it as a cube for simplicity but it can be any shape (Figure \ref{fig_deform}).  If we apply a force that's normal to a surface, we'd expect the fluid element to deform by stretching and thinning.  On the other hand, if the \emph{same} force, in the same direction, is applied to a surface that's parallel to it, the deformation will be very different -- the force in this case is called a ``shear force.''  Evidently, the deformation of a fluid element depends not only on the force direction but also on the \emph{surface} direction. 

\begin{figure}
\centering
\includegraphics[width=0.9\linewidth]{Figures/Chapter2/fig_deform}
\caption{The deformation of a fluid element depends on both the force direction \emph{and} the direction of the surface. }
\label{fig_deform}
\end{figure}

This, unfortunately, adds a complication to how we must describe the forces that deform a fluid element.  There are three orthogonal directions that the force itself could have, and three orthogonal directions that surface it's applied to could have -- that means we need nine different numbers to fully specify things.  Figure \ref{fig_stress_tensor} shows this graphically; the nine elements together are called the \emph{stress tensor} and are labelled $T_{ij}$, where $i$ specifies the direction of the force, and $j$ the direction of the surface.  The elements of the stress tensor are forces per unit area; for example, the element $T_{xy}$ gives the stress, or force per unit area, acting in the $\unit{x}$ direction on a surface with normal in the $\unit{y}$ direciton.

\begin{figure}
\centering
\includegraphics[width=0.7\linewidth]{Figures/Chapter2/fig_stress_tensor}
\caption{The definition of the stress tensor.}
\label{fig_stress_tensor}
\end{figure}

It's common to write the tensors using either ``index notation,'' like $T_{ij}$, or as a matrix:
\[
\tens{T} = \begin{bmatrix}
T_{xx} & T_{xy} & T_{xz} \\
T_{yx} & T_{yy} & T_{yz} \\
T_{zx} & T_{zy} & T_{zz} 
\end{bmatrix}.
\]
Note that the notation for a tensor, $\tens{T}$, is a little different from vectors.  And it turns out that in almost all cases the stress tensor is \emph{symmetric}, so that 
\[
T_{ij} = T_{ji},
\]
which means that we really only have six elements to worry about rather than the full nine.

Suppose we have some surface $S$ with a normal vector $\unit{n}$; the \emph{stress vector} is defined as 
\begin{equation}
\vec{t} = \tens{T} \cdot \unit{n}.
\label{eq_stress}
\end{equation}
Careful here -- that's a dot product between a tensor and a vector, not two vectors, and the result is a vector rather than a scalar.  We can calculate the dot product either by direct summation,
\[
t_i = \sum_{j=1}^3 T_{ij} n_j.
\]
or by matrix multiplicaiton,
\[
\vec{t} = \begin{bmatrix}
T_{xx} & T_{xy} & T_{xz} \\
T_{yx} & T_{yy} & T_{yz} \\
T_{zx} & T_{zy} & T_{zz} 
\end{bmatrix}
\begin{bmatrix}
n_x  \\
n_y  \\
n_z  
\end{bmatrix}.
\]

As a check, let's suppose we have a surface where $\unit{n} = \unit{x}$.  In that case, the stress vector will be
\[
\vec{t} = \begin{bmatrix}
T_{xx} & T_{xy} & T_{xz} \\
T_{yx} & T_{yy} & T_{yz} \\
T_{zx} & T_{zy} & T_{zz} 
\end{bmatrix}
\begin{bmatrix}
1  \\
0  \\
0  
\end{bmatrix} = 
\begin{bmatrix}
T_{xx}  \\
T_{yx}  \\
T_{zx}  
\end{bmatrix},
\]
which makes sense -- the dot product picks out the forces that act on the surface with normal $\unit{x}$.

\subsection{Cauchy's Equation of Motion}

Now we're ready to use Newton's second law to derive a general equation of motion for fluids and other deformable materials.  Consider some ``blob'' of our fluid, with volume $V$ and enclosed by surface $S$; this blob can deform in shape as the fluid moves, but we'll assume our fluid is incompressible, so the total volume can't change.  The force across one small element of area, $da$, due to the fluid above it, is given by the stress vector $\vec{t}$ multiplied by the area $da$.  The total force acting on the entire blob due to the surrounding fluid is therefore
\begin{equation}
\label{eq_stress_tensor_force}
\vec{F} = \int_S \vec{t} \, da = \int_S (\tens{T} \cdot \unit{n}) \, da,
\end{equation}
where I used equation \ref{eq_stress} to write the force in terms of the stress tensor instead of the stress vector.  Now, we can apply the divergence theorem to turn this into an integral over the volume $d\tau$ -- even though this is a tensor, it works pretty much the same way:
\[
\vec{F} =  \int_S (\tens{T} \cdot \unit{n}) \, da = \int_V (\grad \cdot \tens{T} ) \, d\tau.
\]

So far, we've ignored any forces external to the fluid, but we can add them in now.  I'll stick with gravity; for a small volume of the fluid, the force of gravity will be $\rho d\tau \vec{g}$, and we can combine this with the stress force above to get
\[
\vec{F} = \int_V (\grad \cdot \tens{T} + \rho \vec{g} ) \, d\tau.
\]

That's the total force acting on our fluid, and we already know how to write out the acceleration -- that was equation \ref{eq_accel},
\[
\frac{D \vec{u}}{Dt} = \frac{\partial \vec{u}}{\partial t} + (\vec{u} \cdot \vec{\nabla}) \vec{u}.
\]
Newton's second law then gives
\[
\int_V \rho \frac{D \vec{u}}{Dt} \, d\tau = \int_V (\grad \cdot \tens{T} + \rho \vec{g} ) \, d\tau.
\]
But this equation holds for \emph{any} volume of any size within our fluid, so the integrands must be equal:
\begin{equation}
\boxed{
\rho \frac{D \vec{u}}{Dt} =\grad \cdot \tens{T} + \rho \vec{g}.
}
\label{eq_cauchy}
\end{equation}
This is called Cauchy's equation of motion.

\subsection{Newtonian Fluids}

So far we've been very general with the stress tensor; specifying different tensors gives us different fluids modelled with Cauchy's equation.  As a first example of a specific stress tensor -- the simplest possible one! -- consider 
\begin{equation}
T_{ij} = -p \, \delta_{ij},
\end{equation}
where $p(x, y, z, t)$ is a scalar function -- it's the \emph{pressure} in the fluid -- and $\delta_{ij}$ is called the Kronecker delta.  If you're not familiar with it, this is a handy function (technically, it's a tensor) that is \emph{one} if the two indices are the same, and \emph{zero} if they're different:
\[
\delta_{ij} = \begin{cases}
0 & i \neq j, \\
1 & i = j \end{cases}.
\]
As a matrix, this stress tensor therefore looks like
\begin{equation}
\tens{T} = \begin{bmatrix}
-p & 0 & 0 \\
0 & -p & 0 \\
0 & 0 & -p
\end{bmatrix}.
\label{eq_stress_ideal}
\end{equation}
This stress tensor defines \emph{ideal fluids}, which are inviscid (no viscosity) and don't include forces of deformation in the stress tensor.  Why the negative sign on the pressure?  Because this is the force of the fluid \emph{outside} the surface pushing on it -- in the opposite direction of the normal vector $\unit{n}$.

Although ideal fluids have made up a large part of our discussion of fluid dynamics, we're here to derive the Navier-Stokes equations, which deal instead with \emph{Newtonian fluids} and which do include the deformation of fluid elements.  To incorporate that, we'll add a term to the ideal fluid stress tensor:
\begin{equation}
T_{ij} = T_{ij}^I + T_{ij}^D,
\end{equation}
where the $I$ indicates the ideal part and the $D$ indicates the part that deals with the deformation.  

Consider, for a moment, a \emph{solid} instead of a fluid.  If the solid is elastic, then pushing on it will cause it to deform in shape, and the stress that results will be proportional to the change in shape.  This makes sense -- it's just like in Hooke's law, where the force of the spring is proportional to the displacement.  A fluid, though, is very different -- pushing on a fluid causes deformation, but the deformation continues even when the force is stopped.  It turns out that for fluids, the stress is proportional to the \emph{velocities} of the change in shape; that is, rather than
\[
t \propto dx,
\]
we have
\[
t \propto du/dx,
\]
for example.  Note a couple of things here:  first, there are three velocities, $u, v, w$, and three directions, $x, y, z$, so we have nine different velocity gradients we can form.  And second, note that these are derivatives with respect to \emph{position}, not time; these are not accelerations.

Now, the most general symmetric tensor that depends \emph{linearly} on these velocity gradients can be written as
\begin{equation}
T_{ij}^D = \mu \left( \frac{\partial u_j}{\partial x_i} + \frac{\partial u_i}{\partial x_j} \right),
\end{equation}
where $\mu$ is the constant of proportionality and is called (of course) the \emph{viscosity}.  Splitting the velocity gradients into two terms like this ensures the tensor is symmetric.  Although it's cumbersome to do so, and doesn't really gain us anything, I'll write out this tensor in matrix form (you can at least compare with the ideal case in equation \ref{eq_stress_ideal}):
\[
\tens{T}^D = \begin{bmatrix}
2 \mu \partial u / \partial x  &  \mu (\partial v / \partial x + \partial u / \partial y)  &   \mu (\partial w / \partial x + \partial u / \partial z) \\
\mu (\partial u / \partial y + \partial v / \partial x)  & 2\mu  \partial v / \partial y  &  \mu (\partial w / \partial y + \partial v / \partial z) \\
\mu (\partial u / \partial z + \partial w / \partial x)  &  \mu (\partial v / \partial z + \partial w / \partial y) & 2\mu  \partial w / \partial z
\end{bmatrix}.
\label{eq_stress_newtonian}
\]

Adding in the ideal component, the total stress tensor is
\begin{equation}
\label{eq_stress_tensor}
\boxed{
T_{ij} = -p \delta_{ij} +  \mu \left( \frac{\partial u_j}{\partial x_i} + \frac{\partial u_i}{\partial x_j} \right).
}
\end{equation}
This defines a \emph{Newtonian fluid}.  But you should know that there's plenty of examples of \emph{non-Newtonian fluids,} as well -- paint and blood are two good examples.  Non-Newtonian fluids can be nonlinear in the velocity gradients, and can sometimes depend not just on the current state of the fluid, but on the past state as well -- some of these fluids can have a ``memory.''  These kinds of fluids are well outside the scope of this book, however.

\begin{example}[Plane parallel shear flow]
Recall our example from Section \ref{sec_imp} -- we had an impulsively moved boundary, which put the fluid into plane parallel shear flow given in general by $u(y)$.  For simplicity, we'll suppose the boundary is at $y=0$ and the flow exists above it.  What is the stress on the boundary from the fluid?

For flow of this type, the stress tensor becomes
\begin{equation}
\tens{T} = \begin{bmatrix}
-p & \mu \partial u / \partial y & 0 \\
\mu \partial y / \partial y & -p & 0 \\
0 & 0 & -p
\end{bmatrix},
\end{equation}
and we can calculate the stress vector using equation \ref{eq_stress}, with $\unit{n} = \unit{y}$ for the boundary surface:
\[
\vec{t} = \begin{bmatrix}
-p & \mu \partial u / \partial y & 0 \\
\mu \partial y / \partial y & -p & 0 \\
0 & 0 & -p
\end{bmatrix}
\begin{bmatrix}
0 \\ 1 \\ 0
\end{bmatrix}.
\]
Carrying out the matrix multiplication gives
\[
\vec{t} = \mu \frac{\partial u}{\partial y} \unit{x} - p \unit{y}.
\]
This result makes sense: there's a viscous force along the boundary caused by friction with the fluid, and a force pushing downward on the surface from the fluid above.
\end{example}

We're finally ready to derive the Navier-Stokes equation, which is just Cauchy's equation of motion with the stress tensor for Newtonian fluids.  I'm going to write out the calculation in index notation, which is a little easier to deal with, starting with Cauchy's equation,
\[
\rho \frac{Du_i}{Dt} = \sum_{j=1}^3 \frac{\partial T_{ij}}{\partial x_j} + \rho g_i
\]
(compare with the vector version in equation \ref{eq_cauchy}).  The only tricky thing here is the divergence of the stress tensor,
\[
\sum_{j=1}^3 \frac{\partial T_{ij}}{\partial x_j} = \sum_{j=1}^3 \frac{\partial }{\partial x_j} \left[ -p \delta_{ij} + \mu \left( \frac{\partial u_j}{\partial x_i} + \frac{\partial u_i}{\partial x_j} \right) \right],
\]
which is three derivatives.  We'll tackle each separately.

First is the pressure term,
\[
\sum_{j=1}^3 \frac{\partial }{\partial x_j} \left( -p \delta_{ij} \right) = - \sum_{j=1}^3  \frac{\partial p}{\partial x_j} \delta_{ij} = -\frac{\partial p}{\partial x_i}.
\]
Notice that the Kronecker delta \emph{collapsed} the sum -- every term was zero except for the one where $j = i$.  

The next derivative is
\[
\sum_{j=1}^3 \frac{\partial }{\partial x_j} \left( \mu \frac{\partial u_j}{\partial x_i} \right) = \mu \frac{\partial }{\partial x_i} \sum_{j=1}^3 \frac{\partial u_j}{\partial x_j},
\]
where I took the derivative with respect to $x_i$ outside the sum (which is over $j$).  Now, the term in the sum is familiar, even if it's hidden a bit -- it's the divergence of the fluid velocity:
\[
\sum_{j=1}^3 \frac{\partial u_j}{\partial x_j} = \frac{\partial u}{\partial x} + \frac{\partial v}{\partial y} + \frac{\partial w}{\partial z} = \grad \cdot \vec{u}.
\]
But we're assuming an incompressible flow, so the divergence is \emph{zero}, and this derivative goes away.

Finally, we have the third derivative,
\[
\sum_{j=1}^3 \frac{\partial }{\partial x_j} \left( \mu \frac{\partial u_i}{\partial x_j} \right) = \mu \sum_{j=1}^3 \frac{\partial^2 u_i}{\partial x_j^2},
\]
and putting all three together gives us 
\[
\rho \frac{Du_i}{Dt} = -\frac{\partial p}{\partial x_i} + \mu \sum_{j=1}^3 \frac{\partial^2 u_i}{\partial x_j^2} + \rho g_i.
\]
Dividing both sides by the density $\rho$ (and remembering that the kinematic viscosity is $\nu = \mu / \rho$) and using vector notation gives us, at long last, the Navier-Stokes equation,
\begin{equation}
\boxed{
\frac{D \vec{u}}{Dt} = - \frac{1}{\rho} \grad p + \nu \nabla^2 \vec{u} + \vec{g}.
}
\end{equation}


%
% --- SECTION - ADVANCED TOPIC 2 ---
% 

\section{Boundary Layers}
\label{sec_boundary_layers}

The presence of a boundary layer in a fluid is one of the big differences between viscous flow and inviscid.  To explore boundary layers further, let's set up a fairly simple situation.  Consider fluid in the region $y \ge 0$, above a rigid boundary along $y=0$.  Suppose the flow is given by
\begin{equation}
\vec{u} = [\alpha x, -\alpha y, 0],
\end{equation}
where $\alpha$ is some constant.  We've seen this type of flow before, in Chapter 1 -- it's the flow about a stagnation point.  The streamlines are shown in Figure \ref{fig_boundary_stream}.

\begin{figure}
\centering
\includegraphics[width=0.6\linewidth]{Figures/Chapter2/fig_boundary_stream}
\caption{Streamlines for flow about a stagnation point.  The fluid lies above the $x$-axis.}
\label{fig_boundary_stream}
\end{figure}


Now, this flow satisfies the Navier-Stokes equation, with a pressure given by
\begin{equation}
\label{eq_boundary_pressure}
p(x, y) = -\frac{1}{2} \rho \alpha^2 ( x^2 + y^2) + \text{ constant}.
\end{equation}
However, it \emph{doesn't} satisfy the no-slip boundary condition -- the fluid velocity isn't zero at the boundary!  This is actually fine if we're dealing with an inviscid fluid, where we expect the flow to be directed along the boundary, but doesn't work for a viscous fluid.  It also has no indications of a boundary layer.

We need to fix things up, then, to account for the boundary layer and viscosity.  We can do that by changing our flow slightly to
\begin{equation}
\label{eq_boundary_trial}
\vec{u} = [\alpha x f'(\eta), -\sqrt{\nu \alpha} f(\eta), 0],
\end{equation}
where
\[
\eta = \sqrt{ \frac{\alpha}{\nu}} y
\]
is a dimensionless length parameter.  Here, the function $f(\eta)$ is an unknown function that we'll solve for shortly; notice that if $f(\eta) = \eta$, we get back our original flow about a stagnation point.  However, in order to satisfy the no-slip condition and get zero velocity at $y=0$, we need $f \to 0$ \emph{and} $f' \to 0$ when $\eta \to 0$.  Clearly $f(\eta) = \eta$ won't do the trick.  On the other hand, we do expect the flow far away from the boundary to look like our original stagnation point flow; we'll return to this later.

\begin{figure}
\centering
\includegraphics[width=0.7\linewidth]{Figures/Chapter6/fig_boundary_result1}
\caption{The function $f(\eta)$ (black) and $f'(\eta)$ (red).  Note that the boundary layer is clear -- it extends out to around $\eta \sim 2$.  After that, $f(\eta) \approx \eta$. }
\label{fig_boundary_result1}
\end{figure}

To find the function $f(\eta)$ properly, then, we need to make sure that this flow still satisfies the Navier-Stokes equation.  If we plug in the velocity in Equation (\ref{eq_boundary_trial}) into the Navier-Stokes equation, the $x$ and $z$ equations become
\begin{equation}
\label{eq_boundary_ns1}
\alpha^2 x f'^2 - \alpha^2 x f f'' = -\frac{1}{\rho} \frac{\partial p}{\partial x} + \alpha^2 x f'''
\end{equation}
and
\begin{equation}
\sqrt{\alpha^3 \nu } ff' = \frac{1}{\rho} \frac{\partial p}{\partial y} - \sqrt{\alpha^3 \nu} f'',
\end{equation}
respectively.  Let's rearrange the second equation slightly to get
\[
-\frac{1}{\sqrt{\alpha^3 \nu}} \frac{1}{\rho} \frac{\partial p}{\partial y} = \frac{1}{2} \frac{d}{d\eta} f^2 + \frac{d}{d\eta} f'.
\]
We can then replace the $\partial y$ derivative with $\partial \eta$, since
\[
\frac{d\eta}{dy} = \sqrt{\frac{\alpha}{\nu}},
\]
to get
\[
-\frac{1}{\alpha \nu \rho} \frac{\partial p}{\partial \eta} = \frac{1}{2} \frac{d}{d\eta} f^2 + \frac{d}{d\eta} f'.
\]
Finally, integrate with respect to $\eta$:
\[
-\frac{1}{\alpha \nu \rho} p = \frac{1}{2} f^2 + f' + g(x),
\]
where $g(x)$ is the integration ``constant.''  This is a useful result:  if we now take the derivative with respect to $x$, we can see that 
\[
\frac{\partial p}{\partial x} \propto \frac{\partial g}{\partial x}.
\]
In other words, $\partial p / \partial x$ is at best a function of $x$.

\begin{figure}
\centering
\includegraphics[width=\linewidth]{Figures/Chapter6/fig_boundary_result2}
\caption{Vector plots of the invisid flow (left) and viscous flow (right).  The boundary layer can be clearly seen in the flow -- the blue colour indicates small velocity. }
\label{fig_boundary_result2}
\end{figure}

We can now return to the other Navier-Stokes equation, Equation (\ref{eq_boundary_ns1}), and separate variables so that all $x$ terms are on one side and all $\eta$ terms are on the other:
\[
-\frac{1}{\alpha^2 \rho x} \frac{\partial p}{\partial x} = f'^2 - ff'' - f'''.
\]
Since $x$ and $\eta$ (which is really just $y$) are independent, each side of this equation must be a constant.  What is the value of that constant?  To find it, we'll compare with the inviscid solution for the pressure, Equation (\ref{eq_boundary_pressure}).  To have our pressures agree, we need the constant to be one.

Finally, then, we have our last equation we need to solve,
\begin{equation}
f''' + ff'' - f'^2 + 1 = 0.
\end{equation}
As we've discussed above, the boundary conditions are
\[
f(0) = 0; \quad f'(0) = 0; \quad \text{and} \quad f'(\infty) = 1,
\]
where the last condition ensures that we return to our original flow (with $f(\eta) = \eta$) far away from the boundary.

This equation must be solved numerically, but it can be tricky to do so while making sure the boundary conditions match; the results are shown in Figures \ref{fig_boundary_result1} and \ref{fig_boundary_result2}.  As you can see, there is now a small boundary layer where the velocity drops to zero, but away from the boundary the flow is indistinguishable from the inviscid case.  Despite the complexity of this example, it's a good one for exploring exactly how the boundary layer develops for viscous flows.





%
% --- SECTION - ADVANCED TOPIC 3 ---
% 

\section{Nonlinear Waves}

In our discussion of waves, we were careful to focus on those of small-amplitude -- which linearized the equations that describe waves and allowed us to investigate various different properties.  In general, the full theory of waves is quite difficult, but if we limit ourselves to shallow water, we can handle some interesting cases where the amplitude is large.

\subsection{The Shallow Water Equations}

We'll follow the basic set up we had back in Chapter \ref{chap_waves}, but make a few small changes.  First, rather than using $\eta$ for the height of the wave, I'll use $y = h(x, t)$ -- mostly to remind us that these are not small amplitude waves.  And we'll put a flat bottom at $y = 0$; see Figure \ref{fig_nonlinear_wave}.  Furthermore, we won't assume that the amplitude of the wave is small compared to the wavelength.  Instead, we'll make the \emph{shallow water approximation} -- that the depth of the water is much less than typical horizontal scales of the waves themselves.


As with the earlier work on waves, Euler's equation in two dimensions is the starting point, with $\vec{g} = [0, -g, 0]$:
\begin{align}
\frac{Du}{Dt} & = -\frac{1}{\rho} \dfdx{p}{x}, \label{eq_euler_non_u}\\
\frac{Dv}{Dt} & = - \frac{1}{\rho} \dfdx{p}{y} - g. \label{eq_euler_non_v}
\end{align}
But the shallow water approximation suggests we can drop the $Dv/Dt$ term from the second equation -- basically, the acceleration in the vertical direction will be much less than the acceleration due to gravity, and can be neglected.  Equation (\ref{eq_euler_non_v}) then becomes
\[
\dfdx{p}{y} = -\rho g,
\]
which integrates to
\[
p(x, y, t) = -\rho g y + f_1(x, t),
\]
where $f_1(x, t)$ is the integration ``constant'' (which could depend on $x$ or $t$ since we had a partial derivative).  We can find the constant through boundary conditions as usual; in this case we have atmospheric pressure $p_0$ at the surface, $y = h(x, t)$.  This leads to $f_1 = p_0 + \rho g h$, giving us the pressure
\begin{equation}
p(x, y, t) = p_0 - \rho g [ y - h(x, t)].
\end{equation}

\begin{figure}
\centering
\includegraphics[width=0.8\linewidth]{Figures/Chapter6/fig_nonlinear_wave}
\caption{The surface of the fluid is given by $y = h(x, t)$, with the bottom at $y=0$.  These waves can be large in amplitude but the fluid itself is shallow. }
\label{fig_nonlinear_wave}
\end{figure}

If we substitute this pressure into equation (\ref{eq_euler_non_u}), we get
\[
\frac{Du}{Dt} = -g \dfdx{h}{x}.
\]
Now, $h$ is a function of $x$ and $t$ only, which suggests that the \emph{change} in $u$ depends only on those variables.  As long as $u$ doesn't start out depending on $y$, it will never pick up a $y$ dependency as time goes on.  That means we can write $u = u(x, t)$, and, writing out the full material derivative and dropping the $\partial u / \partial y$ term, the above equation becomes
\begin{equation}
\label{eq_shallow1}
\boxed{
\dfdx{u}{t} + u\dfdx{u}{x} = -g \dfdx{h}{x}.
}
\end{equation}

This is the first of two important equations we'll use to describe nonlinear shallow waves.  The second comes from the incompressibility condition, which in two dimensions says
\[
\dfdx{u}{x} + \dfdx{v}{y} = 0.
\]
If we integrate this for $v$, we get
\[
v(x, y, t) = -\dfdx{u}{x} y + f_2(x, t),
\]
where $f_2$ is again an integration constant.  To find this one, we use the boundary condition at the bottom:  we must have $v = 0$ at $y=0$, which means $f_2 = 0$, and
\begin{equation}
\label{eq_nonlin_v}
v = -\dfdx{u}{x} y.
\end{equation}

In addition to this, we can bring in the kinematic condition that we derived back in Section \ref{sec_kin_press_cond}.  Before linearizing, it was (replacing $\eta$ with $h$)
\[
v = \dfdx{h}{t} + u\dfdx{h}{x} \quad \text{at} \quad y = h(x, t).
\]
Combining this with equation (\ref{eq_nonlin_v}) above and rearranging a bit gives
\begin{equation}
\label{eq_shallow2}
\boxed{
\dfdx{h}{t} + u\dfdx{h}{x} + h\dfdx{u}{x} = 0.
}
\end{equation}
Equations (\ref{eq_shallow1}) and (\ref{eq_shallow2}) are called the \emph{shallow water equations}, and are a set of two coupled partial differential equations for the functions $u(x, t)$ (which is the horizontal speed of the fluid) and $h(x, t)$ (which is the height of the surface).  


\subsection{Method of Characteristics}

To solve these shallow water equations, we'll apply a technique called the method of characteristics, which will reduce the two partial differential equations to two ordinary differential equations.  We'll do that by finding a set of \emph{characteristic curves}.

To begin, define a new variable $c(x, t)$ as
\begin{equation}
c(x, t) = \sqrt{gh(x, t)}.
\end{equation}
This new variable is essentially a scaled height of the wave but will help us in making the two shallow water equations more symmetric.  In terms of $c$, they become
\begin{equation}
\dfdx{u}{t} + u\dfdx{u}{x} + 2x \dfdx{c}{x} = 0
\end{equation}
and
\begin{equation}
2\dfdx{c}{t} + 2u\dfdx{c}{x} + c \dfdx{u}{x} = 0.
\end{equation}
If we add these two equations together, we can write the result (with a bit of rearranging) as
\begin{equation}
\label{eq_char_pos}
\left[ \dfdx{}{t} + (u + c) \dfdx{}{x} \right] (u + 2c) = 0.
\end{equation}
Similarly, if we subtract them, we get
\begin{equation}
\label{eq_char_neg}
\left[ \dfdx{}{t} + (u - c) \dfdx{}{x} \right] (u - 2c) = 0.
\end{equation}

These equations look pretty symmetric and are in good shape to use the method of characteristics on.  Here's how that's done -- but be careful, I think this is a tricky argument so try to follow along.

Consider just the positive equation (\ref{eq_char_pos}) above.  Define a parametric curve -- called the \emph{positive characteristic curve} -- in the $x$-$t$ plane using the parameter $\xi$, so that
\[
x = x(\xi) \quad \text{and} \quad t = t(\xi).
\]
The curve will be defined by two properties:  that 
\[
\frac{dt}{d\xi} = 1,
\]
(so that the parameter $\xi$ is proportional to the time) and that
\[
\frac{dx}{d\xi} = u+c.
\]
We'll start the curve at some point $(x_0, t_0)$ (see the blue line in Figure \ref{fig_nonlinear_curves}).

\begin{figure}
\centering
\includegraphics[width=0.7\linewidth]{Figures/Chapter6/fig_nonlinear_curves}
\caption{The positive characteristic curve (blue) is defined by $dx/d\xi = u+c$, and along the curve the quantity $u+2c$ is constant.  The negative characteristic curve (red) is defined by $dx/d\xi = u-c$, and along the curve the quantity $u - 2c$ is constant. }
\label{fig_nonlinear_curves}
\end{figure}

Why this definition?  So that we can write the positive equation (\ref{eq_char_pos}) as
\[
\left[ \frac{dt}{d\xi} \dfdx{}{t} + \frac{dx}{d\xi} \dfdx{}{x} \right] (u + 2c) = 0.
\]
Now the term in the square brackets looks like a chain rule, meaning we can write this as simply
\begin{equation}
\boxed{
\frac{d}{d\xi} (u + 2c) = 0.
}
\end{equation}
What does this mean?  As the positive characteristic curve is traced out, the quantity $u+2c$ stays \emph{constant}; see the blue curve in Figure \ref{fig_nonlinear_curves}.  Notice that we've reduced our original partial differential equation to an ordinary one; the penalty is that this equation is true only along the curve defined by $dx/d\xi = u+c$.

The whole argument above can be repeated with the negative equation (\ref{eq_char_neg}), leading to a characteristic curve defined by $dx/d\xi = u-c$, and along this curve
\begin{equation}
\boxed{
\frac{d}{d\xi} (u - 2c) = 0.
}
\end{equation}

Although the details at this point might be a little difficult to follow, the method of characteristics can be a powerful tool.  A classic example should illustrate what's going on nicely.

\subsection{The Dam Break Problem}

Suppose a large dam holds back water of height $h_0$ as shown in Figure \ref{fig_dam_start}.  The dam is located at $x=0$, and the water is to the left, but at $t=0$, the dam breaks -- in our case, that means it disappears instantly -- and the water can rush out to the right.  We want to find two things: the horizontal speed of the water $u(x, t)$ and its height $h(x, t)$.  This is a famous problem in fluid dynamics, not least because of the deadly and destructive nature of dam breaks, and was first solved in 1892 by Ritter.

\begin{figure}
\centering
\includegraphics[width=0.9\linewidth]{Figures/Chapter6/fig_dam_start}
\caption{A dam at $x=0$ holds back water of a height $h_0$.  At $t=0$, the dam breaks and the water rushes forward.}
\label{fig_dam_start}
\end{figure}

Consider the state of the water \emph{before} the dam breaks:  for $x<0$ and $t<0$.  Since the water is at rest, we know that $u = 0$ and $c = \sqrt{gh_0} = c_0$.  We can start to build the characteristic curves with this information; let's do the positive curve first, defined by $dx/d\xi = u+c$.  In this region, then, we have
\[
\frac{dx}{d\xi} = c_0,
\]
or, since $dt/d\xi = 1$, 
\[
\frac{dx}{dt} = c_0.
\]
Thus the positive characteristic curves in $x<0$, $t<0$ are straight lines of slope $c_0 = \sqrt{gh_0}$.  I've plotted a few of these curves in Figure \ref{fig_char_curves} (a) (the blue lines); each has the same slope but different intercepts.  We can use similar reasoning for the negative characteristic curves to find that they have a slope of $-c_0$ (the red lines).

Now, we know that along the positive curves the quantity $u + 2c$ is constant.  But in this region we know that $u + 2c = 2c_0$ since $u=0$; similarly for the negative curves: $u - 2c = -2c_0$ along them.  Here's the trick: this is true even for those curves that extend \emph{beyond} the region $x<0, t<0$.

\begin{figure}
\centering
\includegraphics[width=\linewidth]{Figures/Chapter6/fig_char_curves}
\caption{Constructing the characteristic curves for the dam break problem. }
\label{fig_char_curves}
\end{figure}

For example, take a look at point P$_1$ in Figure \ref{fig_char_curves} (a) -- it happens to be at an intersection of the two curves shown as dashed lines, one positive and one negative.  Now, I don't know that those characteristic curves are still straight lines -- when $t>0$ the fluid has started to move, so our analysis above no longer holds.  But we do know that along the positive line 
\[
u + 2c = 2c_0
\]
and along the negative
\[
u - 2c = -2c_0.
\]
Right at point P$_1$, both of these conditions are true, and we can rearrange and solve for $u$ and $c$ there.  It turns out that
\[
u = 0 \quad \text{and} \quad c = c_0
\]
at point P$_1$ -- but that means the lines \emph{are} straight there, and will be throughout the region where the positive and negative curves intersect.

That's where the next twist comes in -- the negative curves can't cover then entire second quadrant, since they must start from the third where the fluid is at rest.  So the last negative characteristic curve is given by
\[
x = -c_0 t,
\]
and all curves below this line -- the thick red line in Figure \ref{fig_char_curves} (b) -- are straight.  That means for positions less than $-c_0 t$ the fluid is \emph{still} at rest; this is to the left of the original dam location, and evidently the knowledge that the dam is no longer there has not yet been communicated beyond this point and the water there is undisturbed.

Beyond this line, though, things get a little more complicated.  Consider the point P$_2$ in Figure \ref{fig_char_curves} (b).  It lies along a positive characteristic curve that originally started in the third quadrant, so even though this line is no longer straight, is still has
\[
u + 2c = 2c_0
\]
along it.  There will be some negative curve passing through this point as well -- but we don't know anything about this one other than $u-2c$ is constant along it.  Let's call the constant $k$, so that
\[
u - 2c = k
\]
along this negative curve.  At point P$_2$, then, we can use these two equations to solve for $u$ and $c$; they are
\[
u = c_0 + \frac{k}{2} \quad \text{and} \quad c = \frac{c_0}{2} - \frac{k}{4}.
\]
The important thing to note is that along the negative characteristic curve, both $u$ and $c$ are constant (since $k$ is constant along that curve; the same is \emph{not} true for the positive curve).  Once again we find that the negative characteristic curves are straight lines, this time with slope
\[
\frac{dx}{dt} = u - c = \frac{c_0}{2} + \frac{k}{4}.
\]

That's great, but -- what do these straight lines actually look like?  Well, with a bit of thought you might realize that the negative characteristic curves can never cross (and the same is true of the positive, although obviously positive and negative can cross).  If they did intersect at some point, the water there would have two different values of $u$ or $c$.  Normally the velocity and height are single-valued functions, so this is not allowed -- except at one point: at $x = 0$, $t=0$ we have a discontinuity in the height of the water.  Thus the negative characteristic curves \emph{can} cross at that point.  So although we can't specify the slope of these lines, we do know that they start at the origin.  This is shown in \ref{fig_char_curves} (c). 

Let's quickly recap what we know in the region above the line $x = -c_0 t$.  Along the negative curves, 
\[
\frac{dx}{dt} = u - c = \text{constant},
\]
which we can integrate to get
\[
x = (u-c) t
\]
(the integration constant is zero since the line passes through the origin).
Along the positive curves, we have 
\[
u + 2c = c_0.
\]
If we rearrange these two equations are solve for $u$ and $c$ we get
\begin{equation}
\label{eq_dam_dist}
u(x, t) = \frac{2c_0}{3} + \frac{2x}{3t} \quad \text{and} \quad c(x, t) = \frac{2c_0}{3} - \frac{x}{3t}.
\end{equation}
This is the velocity and height in the region $x>-c_0 t$.  But note that $c = \sqrt{gh}$ can never go negative -- so we must have, from the above equation
\[
\frac{2c_0}{3} > \frac{x}{3t},
\]
or $x < 2c_0 t$.  This gives us the furthest extent of the water rushing from the dam -- as time goes on, the water travels further.  This is shown as the thick red line in Figure \ref{fig_char_curves} (c) and (d).

We're all finished at this point -- the fluid is either undisturbed or described by equation (\ref{eq_dam_dist}) -- but before summarizing, let's finish off the positive characteristic curves in the region $x>-c_0 t$.  The slope of these curves is given by
\[
\frac{dx}{dt} = u+c = \frac{4c_0}{3} + \frac{x}{3t},
\]
and we can solve this differential equation to get
\begin{equation}
x(t) = c_1 x^{1/3} + 2c_0 x,
\end{equation}
where $c_1$ is an integration constant and varies from positive curve to positive curve.  These lines are shown in Figure \ref{fig_char_curves} (d) and complete the characteristic curve diagram for the dam break problem.

In summary, then, the height of the surface of the water is given by
\begin{equation}
h(x, t) = \begin{cases}
h_0, & x < -\sqrt{gh_0} t \\
\frac{1}{g} \left[ \frac{2}{3} \sqrt{gh_0} - \frac{x}{3t} \right]^2, &  -\sqrt{gh_0} t < x < 2\sqrt{gh_0} t \\
0, & x > 2\sqrt{gh_0} t,
\end{cases}
\end{equation}
and the velocity is
\begin{equation}
u(x, t) = \begin{cases}
0, & x < -\sqrt{gh_0} t \\
\frac{2}{3} \sqrt{gh_0} + \frac{2x}{3t}, & -\sqrt{gh_0} t < x < 2\sqrt{gh_0} t.
\end{cases}
\end{equation}
The surface is plotted in Figure \ref{fig_dam_heights} for two different times.  

\begin{figure}
\centering
\includegraphics[width=0.85\linewidth]{Figures/Chapter6/fig_dam_heights}
\caption{The height profile $h(x, t)$ for the water from a dam break at two different times.  Distance units at in meters; at 2 s, the water has already reached 30 m from the original dam location.}
\label{fig_dam_heights}
\end{figure}

Notice that the speeds for typical modern dam heights can get very large; in Figure \ref{fig_dam_heights}, I used a height of $h_0 = 10$ m, and at 2 s after the break, the leading edge of the water ($x = 2c_0 t \sim 40$ m) is travelling around 20 m/s.  For larger dams, which can have heights of well over 100 m, the speed is in the hundreds of kilometers per hour.




%
% --- SECTION - ADVANCED TOPIC 4 ---
% 


\section{Flow Past a Sphere}

Throughout almost the entire book we've dealt only with flows that are two dimensional, but there's a classic problem involving flow around a \emph{sphere} -- decidedly not a two dimensional object -- that's worth tackling.  We'll start with the simpler case of ideal fluid flowing around a sphere, which is very similar to the cylinder problem we've already done in Section \ref{sec_cylinder}, and build up to the flow of very viscous fluid past a sphere.

\subsection{Ideal Flow}
\label{sec_sphere_ideal}

Consider first, then, the ideal flow around a sphere of radius $a$.  The fluid starts out, far away from the sphere, in uniform flow with speed $U$.  We'll put the sphere at the origin and arrange our coordinate system such that the uniform flow is along the $z$ axis, so far away $\vec{u} = [0, 0, U]$; see Figure \ref{fig_sphere_setup}.

\begin{figure}
\centering
\includegraphics[width=0.8\linewidth]{Figures/Chapter6/fig_sphere_setup}
\caption{Fluid that is uniform at infinity flows past a sphere.}
\label{fig_sphere_setup}
\end{figure}

Now, this flow is \emph{irrotational} for the same reason ideal flow about a cylinder is -- since the flow is uniform at infinity, it obviously has zero vorticity there, and Kelvin's circulation theorem (Section \ref{sec_circulation}) says that it will remain irrotational as it flows past the sphere.  That means we can describe the flow with a velocity potential $\varphi$ as usual and, since the flow is incompressible, it must satisfy Laplace's equation,
\[
\nabla^2 \varphi = 0.
\]

Before we start the heavy machinery of solving Laplace's equation via separation of variables, though, it's worthwhile to look at two things:  whether there's any symmetry in the problem we can exploit, and what our boundary conditions are.  We \emph{do} in fact have symmetry here:  since we could rotate the flow and sphere by any angle we wish around the $z$ axis and not change the problem, there must not be any dependence on the spherical $\phi$ coordinate (that's the azimuthal angle, measured from the $x$ axis around the $z$ axis).  This is called \emph{axisymmetry}.  In spherical coordinates, then, we can write
\begin{equation}
\vec{u} = \vec{u}(r, \theta),
\end{equation}
and write Laplace's equation as
\begin{equation}
\label{eq_laplace_sph}
\nabla^2 \varphi = \frac{1}{r^2} \dfdx{}{r} \left( r^2 \dfdx{\varphi}{r} \right) + \frac{1}{r^2 \sin \theta} \dfdx{}{\theta} \left(\sin \theta \dfdx{\varphi}{\theta} \right).
\end{equation}

The boundary conditions turn out to be similar to those for flow around a cylinder (Section \ref{sec_cylinder} again).  We need the fluid to slip along the boundary of the sphere, which means there can't be a radial component there, so the first boundary condition is
\begin{equation}
\label{eq_sphere_bc1}
u_r(a, \theta) = 0 \quad \text{or} \quad \left. \dfdx{\varphi}{r} \right|_{r=a} = 0.
\end{equation}
The second condition is that the flow far away from the sphere must be uniform along the $z$ direction.  In spherical coordinates, this uniform flow is given by
\begin{equation}
\label{eq_sphere_uni}
u_r(r, \theta) = U\cos \theta \quad \text{and} \quad u_\theta(r, \theta) = -U \sin \theta.
\end{equation}
From $\vec{u} = \grad \varphi$, which in spherical coordinates is
\[
u_r = \dfdx{\varphi}{r} \quad \text{and} \quad u_\theta = \frac{1}{r} \dfdx{\varphi}{\theta},
\]
we get that the velocity potential for uniform flow is 
\[
\varphi_\text{uni} = U r \cos \theta.
\]
The boundary condition is therefore
\begin{equation}
\label{eq_sphere_bc2}
\varphi(r, \theta) \to Ur \cos \theta \quad \text{as} \quad r \to \infty.
\end{equation}

Laplace's equation in spherical coordinates with axisymmetry is a well-studied problem in numerous physics subjects; the angular part of the solution is the Legendre polynomials $P_\ell (\cos\theta)$.  However, we can skip the full separation of variables if we examine that second boundary condition and suppose that our solution must take on the form
\begin{equation}
\label{eq_pot_form}
\varphi(r, \theta) = f(r) \cos \theta,
\end{equation}
where $f(r)$ is some unknown function.  This is just the $\ell = 1$ Legendre polynomial.  

We can find $f(r)$ by plugging equation (\ref{eq_pot_form}) into Laplace's equation as given in equation (\ref{eq_laplace_sph}).  After some work, it reduces to the ordinary differential equation
\begin{equation}
r^2 \frac{d^2f}{dr^2} + 2r \frac{df}{dr} - 2f = 0.
\end{equation}
This equation can be solved by guessing that the solution is a power-law, $f(r) = r^m$.  Plugging this form in gives us the quadratic
\[
m^2 + m - 2 = 0,
\]
which has solutions $m = 1$ and $m = -2$.  Thus the general solution is
\[
f(r) = Ar + \frac{B}{r^2},
\]
and the velocity potential is then
\[
\varphi(r, \theta) = \left(Ar + \frac{B}{r^2} \right) \cos \theta.
\]

Fitting the boundary conditions, equations (\ref{eq_sphere_bc1}) and (\ref{eq_sphere_bc2}), gives
\[
A = U \quad \text{and} \quad B = \frac{Ua^3}{2},
\]
so the final velocity potential for ideal flow around a sphere is
\begin{equation}
\label{eq_sphere_pot}
\varphi(r, \theta) = U \left( r + \frac{a^3}{2r^2} \right) \cos \theta.
\end{equation}

Given the potential, we can find the velocity $\vec{u}$ by taking the gradient; Problem \ref{prob_sphere} asks you to do it, and the result is
\begin{equation}
u_r(r, \theta) = U \left( 1 - \frac{a^3}{r^3} \right) \cos \theta
\end{equation}
and
\begin{equation}
u_\theta(r, \theta) = -U \left( 1 + \frac{a^3}{2r^3} \right) \sin \theta.
\end{equation}
Feel free to compare this solution with that for the cylinder in Section \ref{sec_cylinder}.  

We could, at this point, calculate the pressure on the sphere and find the force that the fluid exerts on it.  However, I'll skip the details here -- I've shown the pressure throughout the fluid in Figure \ref{fig_sphere_lines} if you're interested -- and just mention that the end result is exactly the same as in the cylinder case:  the net force is zero.  This is just d'Alembert's paradox showing up again since we'll dealing with an ideal fluid.

\subsection{Stokes Stream Function}

How do we plot the streamlines for the fluid?  For two dimensional, incompressible, irrotational flow we had the stream function $\psi$, which was constant on streamlines and made it simple to plot them.  For axisymmetric flow, the same stream function no longer works.  However, there is an analog for this case, called \emph{Stokes stream function}\footnote{The two dimensional stream function $\psi$ is then sometimes called the Lagrange stream function.  Then again, some places call both versions Stokes stream functions, or just stream functions, so be careful.} $\Psi$, defined such that
\begin{equation}
\vec{u} = \grad \times \left( \frac{1}{r\sin \theta} \Psi \unit{\phi} \right).
\end{equation}
In spherical coordinates, this definition becomes
\begin{equation}
\label{eq_stokes_stream}
u_r = \frac{1}{r^2 \sin \theta} \dfdx{\Psi}{\theta} \quad \text{and} \quad u_\theta = -\frac{1}{r \sin \theta} \dfdx{\Psi}{r}.
\end{equation}

You can show for yourself (Problem \ref{prob_sphere}) that with this definition and for axisymmetric flow, 
\[
(\vec{u} \cdot \grad) \Psi = 0,
\]
so that $\Psi$ is indeed constant along streamlines as we wanted.  

For the sphere, you can show that (Problem \ref{prob_sphere} again)
\begin{equation}
\Psi (r, \theta) = \frac{1}{2} U \left( r^2 - \frac{a^3}{r} \right) \sin^2 \theta.
\end{equation}
The streamlines are plotted in Figure \ref{fig_sphere_lines}; again, you might want to compare with the streamlines for flow around a cylinder in Figure \ref{fig_cyl_streamlines}.

\begin{figure}
\centering
\includegraphics[width=0.8\linewidth]{Figures/Chapter6/fig_sphere_lines}
\caption{Streamlines for ideal flow about a sphere.  The pressure is shown as well, with red regions indicating high pressure and blue low.}
\label{fig_sphere_lines}
\end{figure}

\subsection{Very Viscous Flow}

Suppose instead of ideal fluid flowing past the sphere, we have Newtonian viscous flow instead.  If the flow is \emph{steady} and we ignore the effect of gravity, the Navier-Stokes equation (\ref{eq_navier_stokes}) is then
\begin{equation}
\label{eq_ns_steady}
(\vec{u} \cdot \grad ) \vec{u} = -\frac{1}{\rho} \grad p + \nu \nabla^2 \vec{u}.
\end{equation}

There are two terms involving velocity in this equation, and their relative strength determines how a fluid will flow.  We've already studied at length flows for which the \emph{inertial term} $(\vec{u} \cdot \grad ) \vec{u}$ is dominant over the \emph{viscous term} $\nu \nabla^2 \vec{u}$ -- that's just ideal fluid flow.  Here we turn to the case where the viscous term is much larger than the inertial, in which case we'll drop that term and equation (\ref{eq_ns_steady}) simply becomes
\begin{equation}
\label{eq_ns_slow}
\grad p = \mu \nabla^2 \vec{u}
\end{equation}
(notice that this has the dynamic viscosity $\mu = \nu / \rho$ in it).  This equation is of course supplemented as always by the incompressibility condition,
\begin{equation}
\grad \cdot \vec{u}.
\end{equation}
Together, these equations are sometimes called the \emph{slow flow equations} -- they are a good description of fluid flow that is \emph{very viscous} or, alternatively, has a small Reynolds number,
\[
R = \frac{UL}{\nu} \ll 1.
\]

Before we proceed with slow flow past a sphere, I want to rewrite equation (\ref{eq_ns_slow}) to get rid of the pressure term.  We can do that by using the vector identity 
\[
\nabla^2 \vec{u} = \grad (\grad \cdot \vec{u}) - \grad \times (\grad \times \vec{u}).
\]
Plugging this into the slow flow equation (\ref{eq_ns_slow}), dropping the first term (sine we have incompressible flow), and writing it in terms of the vorticity $\vec{\omega} = \grad \times \vec{u}$, we get
\begin{equation}
\label{eq_ns_slow_p}
\grad p =  -\mu \grad \times \vec{\omega}.
\end{equation}
Finally, if we take the curl of both sides, the left hand side becomes zero (since the curl of a gradient is identically zero; see Problem \ref{prob_vc2}), and we get
\begin{equation}
\label{eq_slow_flow_curl}
\boxed{
\grad \times (\grad \times \vec{\omega}) = \vec{0}.
}
\end{equation}
This is the equation we'll solve; it actually looks simpler than it is, since in terms of the velocity $\vec{u}$ it has \emph{three} curls.

\subsection{Slow Flow Past a Sphere}

Now we can attempt a solution for the case of very viscous flow past a sphere.  This is a classic problem in fluid dynamics, first done by Stokes in 1851.  It's not too difficult, but it \emph{is} lengthy and a little complicated, so I'll go slow and carefully.

The set up will look just like the ideal case as shown in Figure \ref{fig_sphere_setup}: the fluid will be uniform at infinity with speed $U$, and our coordinate system will be such that the flow is along the $z$ direction and the sphere of radius $a$ is at the origin.  So, once again we have axisymmetry, and the velocity will not depend on $\phi$ at all:
\[
\vec{u} = u_r(r, \theta) \unit{r} + u_\theta(r, \theta) \unit{\theta}.
\]

Unfortunately, though, this problem is no longer irrotational, so Laplace's equation is no longer appropriate, and our similarity to the ideal case stops.  However, we \emph{do} still have incompressible, axisymmetric flow, so Stokes stream function $\Psi(r, \theta)$ can be used to describe the fluid.  Equation (\ref{eq_stokes_stream}) gives the velocity in terms of $\Psi$, and I'll rewrite it here so you don't have to flip back to see it:
\[
u_r = \frac{1}{r^2 \sin \theta} \dfdx{\Psi}{\theta} \quad \text{and} \quad u_\theta = -\frac{1}{r \sin \theta} \dfdx{\Psi}{r}.
\]

\subsubsection{The Equation of Motion}

Our first goal is to write the slow flow equation (\ref{eq_slow_flow_curl}) in terms of the stream function $\Psi$.  That's a lot of curls we have to work through, but the end result isn't too bad.

We can calculate the vorticity easily enough; since there's no $u_\phi$ or any $\phi$ dependence, the curl of $\vec{u}$ will only have a component in the $\unit{\phi}$ direction.  In spherical coordinates, this is
\begin{equation}
\vec{\omega} = \grad \times \vec{u} = \frac{1}{r} \left[ \dfdx{(r u_\theta)}{r}  - \dfdx{u_r}{\theta} \right] \, \unit{\phi}.
\end{equation}
Plugging in $u_r$ and $u_\theta$ from above, this becomes, in terms of $\Psi$,
\[
\vec{\omega} = -\frac{1}{r\sin\theta} \left[ \ddfdx{\Psi}{r} + \frac{\sin\theta}{r^2} \dfdx{}{\theta} \left( \frac{1}{\sin\theta} \dfdx{\Psi}{\theta} \right) \right] \, \unit{\phi}.
\]
Now, this is a complicated expression, so we're going to use a little trick to make the subsequent work easier.  It's not much of a trick: define an operator $\LL$ as 
\begin{equation}
\LL = \ddfdx{}{r} + \frac{\sin\theta}{r^2} \dfdx{}{\theta} \left( \frac{1}{\sin\theta} \dfdx{}{\theta} \right),
\end{equation}
so that the vorticity becomes
\[
\vec{\omega} = -\frac{1}{\sin\theta} \, \LL \Psi \, \unit{\phi}.
\]
We're just hiding some complexity inside of that new operator, but it turns out that this pattern will repeat.

Next we need to take the curl of the vorticity.  With some tedious but not difficult work, you can show that it is
\begin{equation}
\label{eq_curl_vort}
\grad \times \vec{\omega} = -\frac{1}{r^2 \sin\theta} \dfdx{}{\theta} (\LL \Psi) \, \unit{r} + \frac{1}{r\sin\theta} \dfdx{}{r} ( \LL\Psi) \, \unit{\theta}.
\end{equation}
Now we just need one more curl.  Since $\grad \times \vec{\omega}$ only has $r$ and $\theta$ directions and dependence, once again the curl will only have a $\unit{\phi}$ direction.  It is
\[
\grad \times \grad \times \vec{\omega} = \frac{1}{r \sin\theta} \left[ \ddfdx{}{r} (\LL \Psi) + \frac{\sin \theta}{r^2} \dfdx{}{\theta} \left( \frac{1}{\sin\theta} \dfdx{}{\theta} (\LL \Psi) \right) \right] \, \unit{\phi}.
\]
See the pattern?  This is just
\[
\grad \times \grad \times \vec{\omega} = \frac{1}{r \sin\theta} \LL (\LL \Psi) \, \unit{\phi},
\]
or, extending the notation in an obvious way,
\[
\grad \times \grad \times \vec{\omega} = \frac{1}{r \sin\theta} \LLLL \Psi \, \unit{\phi}.
\]

We have, then, our equation of motion for very viscous fluid moving past a sphere; plugging this into equation (\ref{eq_slow_flow_curl}), we have
\begin{equation}
\label{eq_slow_flow_psi}
\boxed{
\LLLL \Psi = 0.
}
\end{equation}
This might \emph{look} simple, but don't be fooled:  this is a fourth order partial differential equation.


\subsubsection{The Boundary Conditions}

As with the ideal case above, looking at the boundary conditions can give us a hint on how to proceed with actually solving this fourth order equation.  The boundary conditions are actually pretty similar to that ideal fluid problem.

First, we need to apply the no-slip condition to the surface of the sphere -- the fluid must have zero velocity there, so
\[
u_r(a, \theta) = 0 \quad \text{and} \quad u_\theta(a, \theta) = 0.
\]
We can write this condition in terms of the stream function $\Psi$ using equation (\ref{eq_stokes_stream}) and get
\begin{equation}
\label{eq_slow_bc1}
\dfdx{\Psi}{\theta} \bigg|_{r=a} = 0 \quad \text{and} \quad \dfdx{\Psi}{r} \bigg|_{r=a} = 0.
\end{equation}

Second, the flow needs to be uniform in the $z$ direction far away from the sphere.  That means we need
\[
\vec{u}(r \to \infty) = U \, \unit{z},
\]
or, in spherical coordinates,
\[
u_r \to U\cos\theta \quad \text{and} \quad u_\theta \to  - U \sin\theta \quad \text{as} \quad r \to \infty.
\]
In terms of the Stokes stream function, this boundary condition is
\begin{equation}
\label{eq_slow_bc2}
\Psi \to \frac{1}{2} U r^2 \sin^2 \theta.
\end{equation}

\subsubsection{The Solution}

This second boundary condition suggests that our solution will take the form
\begin{equation}
\Psi(r, \theta) = f(r) \sin^2\theta.
\end{equation}
This form of $\Psi$ will simplify things quite a bit.  Let's start by applying the operator $\LL$ to this form of $\Psi$; after a few lines of work, I get
\[
\LL \Psi = \left( f'' - \frac{2}{r^2} f \right) \sin^2\theta,
\]
where the double prime indicates two derivatives with respect to $r$.

We can make the next step a little easier if we define a new function 
\begin{equation}
\label{eq_sfs_F}
F(r) = f'' - \frac{2}{r^2} f.
\end{equation}
Then applying the $\LL$ operator again obviously gives us the same result:
\[
\LL (\LL \Psi) =  \left( F'' - \frac{2}{r^2} F \right) \sin^2\theta,
\]
just in terms of $F$ rather then $f$.  Equation (\ref{eq_slow_flow_psi}),
\[
\LLLL \Psi = 0,
\]
becomes 
\[
F'' - \frac{2}{r^2} F = 0.
\]
Plugging in equation (\ref{eq_sfs_F}) gives us 
\begin{equation}
f^{\prime\prime\prime\prime} - \frac{4}{r^2} f'' + \frac{8}{r^3} f' - \frac{8}{r^4} f = 0.
\end{equation}

As promised, this is a fourth order differential equation that we need to solve.  However, this is easier than you might guess; first, multiply by $r^4$ to get
\[
r^4 \frac{d^4f}{dr^4} - 4r^2 \frac{d^2f}{dr^2} + 8r \frac{df}{dr} - 8f = 0.
\]
Notice that each term is paired with a similar order $r$ and derivative; this suggests we try a power-law solution of the form 
\[
f(r) = r^m.
\]
Plugging this into the differential equation above gives us a fourth-order polynomial,
\[
(m-3)(m-2)(m-1)m - 4(m-1)m + 8m - 8 = 0,
\]
which has the four roots
\[
m = -1, 1, 2, \text{ and } 4.
\]
Thus the general solution for $f(r)$ is
\begin{equation}
f(r) = \frac{A}{r} + Br + Cr^2 + Dr^4.
\end{equation}
Now we just have to find the constants $A, B, C,$ and $D$ using the boundary conditions.

Let's start with the second, that the flow must be uniform at infinity -- so $\Psi \sim r^2$ there.  Of the four terms in the general solution, it doesn't really tell us anything about the first two, since the first ($A$) term goes to zero at infinity, and the second ($B$) doesn't get as big as $r^2$ so it can be neglected as $r\to\infty$.  But it does tell us that we need to drop the fourth term by setting $D = 0$, since that term blows up faster than $r^2$.  Then, looking at the form of the stream function in equation (\ref{eq_slow_bc2}), we must have for the third term
\[
C = \frac{1}{2}U.
\]

The first boundary condition, equation (\ref{eq_slow_bc1}), is for the derivatives of $\Psi$ to go to zero at $r=a$.  I'll leave it to you to work out the derivatives themselves and show that
\[
A = \frac{1}{4} Ua^3 \quad \text{and} \quad B = -\frac{3}{4} U a.
\]

Finally, after much work, we come to our solution at last:  Stokes stream function for slow flow past a sphere is
\begin{equation}
\boxed{
\Psi(r, \theta) = U \left( \frac{a^3}{4r} - \frac{3a}{4} r + \frac{1}{2} r^2 \right) \sin^2\theta.
}
\end{equation}
From this we can immediately plot the streamlines of the flow (Figure \ref{fig_sphere_viscous_lines}),  and we can find the velocities from equation (\ref{eq_stokes_stream}),
\begin{equation}
\label{eq_sphere_viscous_vel1}
u_r(r, \theta) = U \left( \frac{a^3}{2r^3} - \frac{3a}{2r} + 1 \right) \cos\theta,
\end{equation}
and
\begin{equation}
\label{eq_sphere_viscous_vel2}
u_\theta(r, \theta) = U \left( \frac{a^3}{4r^3} + \frac{3a}{4r} - 1 \right) \sin\theta.
\end{equation}


\begin{figure}
\centering
\includegraphics[width=0.8\linewidth]{Figures/Chapter6/fig_sphere_viscous_lines}
\caption{Streamlines for very viscous flow past a sphere.  The colours indicate pressure, with red areas being high pressure.}
\label{fig_sphere_viscous_lines}
\end{figure}

\subsection{Stokes Law}

More interesting in this case is the \emph{force} on the sphere from the fluid.  In our previous examples of ideal fluid flow past objects, the force was always zero, called d'Alembert's paradox.  But now that we have actual viscous fluid, there \emph{will} be a force.  Our goal is to calculate it, but be warned:  we could get away with just integrating over the pressure in past examples, but with viscous fluids there are additional forces to take into account.

\subsubsection{The Pressure}

We'll start with the pressure, though, since we have an simple expression for it in equation (\ref{eq_ns_slow_p}),
\[
\grad p = -\mu \grad \times \vec{\omega}.
\]
In spherical coordinates (and ignoring the $\phi$ component since we have axisymmetry) this is two equations,
\begin{align}
\dfdx{p}{r} & = -\mu [\grad \times \vec{\omega}]_r, \\
\frac{1}{r}\dfdx{p}{\theta} & = -\mu [\grad \times \vec{\omega}]_\theta,
\end{align}
where I'm sorry for the poor notation indicating the components of $\grad \times \vec{\omega}$.  Those components are easy to find from equation (\ref{eq_curl_vort}), where
\[
\LL \Psi = \frac{3Ua}{2r} \sin^2\theta.
\]
So 
\[
[\grad \times \vec{\omega}]_r = -\frac{3Ua}{r^3} \cos\theta
\]
and
\[
[\grad \times \vec{\omega}]_\theta = -\frac{3Ua}{2r^3} \sin\theta,
\]
and the two pressure equations become
\begin{align}
\dfdx{p}{r} & =  \frac{3U\mu a}{r^3} \cos\theta, \\
\frac{1}{r}\dfdx{p}{\theta} & = \frac{3U\mu a}{2r^3} \sin\theta.
\end{align}


Integrating gives the pressure,
\begin{equation}
\label{eq_sphere_slow_p}
\boxed{
p(r, \theta) = p_\infty - \frac{3U \mu a}{2r^2} \cos \theta,
}
\end{equation}
where $p_\infty$ is the integration constant and is the pressure at infinity.  Figure \ref{fig_sphere_viscous_lines} shows the pressure field in the fluid surrounding the sphere, where red indicates high pressure and blue low.  Of course, for the force calculation we'll need the pressure evaluated at the surface of the sphere $r=a$; this is
\begin{equation}
\label{eq_sphere_p_a}
p(a, \theta) = p_\infty -\frac{3U\mu}{2a} \cos \theta.
\end{equation}
As you can see in Figure \ref{fig_sphere_viscous_lines}, the pressure is a maximum at $\theta = \pi$ and a minimum when $\theta = 0$.  This pressure difference upstream and downstream of the flow accounts for part of the force on the sphere (but not all of it).  You should compare with Figure \ref{fig_cyl_pressure} for ideal flow past a cylinder, for which there is no force at all.

\subsubsection{The Stress}

In Section \ref{sec_ns_deriv} I introduced the stress tensor $\tens{T}$, which is a collection of nine different quantities $T_{ij}$ that quantify the force exerted by the fluid, including both the pressure and viscous forces.  For Newtonian fluids, this stress tensor can be written as (see equation \ref{eq_stress_tensor})
\[
\tens{T} = \begin{bmatrix}
- p + 2 \mu \partial u / \partial x  &  \mu (\partial v / \partial x + \partial u / \partial y)  &   \mu (\partial w / \partial x + \partial u / \partial z) \\
\mu (\partial u / \partial y + \partial v / \partial x)  & -p + 2\mu  \partial v / \partial y  &  \mu (\partial w / \partial y + \partial v / \partial z) \\
\mu (\partial u / \partial z + \partial w / \partial x)  &  \mu (\partial v / \partial z + \partial w / \partial y) & -p + 2\mu  \partial w / \partial z
\end{bmatrix}.
\]

Unfortunately, this is in Cartesian coordinates, and we're working with spherical -- and the corresponding tensor in spherical coordinates is very messy.  Luckily, as we'll see, we really only need two different components of the stress tensor:  $T_{rr}$, the force in the radial direction against a surface with normal in the radial direction, and $T_{\theta r}$, the force in the $\unit{\theta}$ direction against a surface with normal in the radial direction.  Deriving these two components is a little beyond the scope of this book,\footnote{I looked them up for this calculation; see, for example, \emph{Elementary Fluid Dynamics} by D. J. Acheson, appendix A.7.} but I can give them to you here:
\begin{align}
T_{rr} & = -p + 2\mu \dfdx{u_r}{r} \\
T_{\theta r} & = \mu \left( \frac{1}{r} \dfdx{u_r}{\theta} + \dfdx{u_\theta}{r} - \frac{u_\theta}{r} \right).
\end{align}

For our case, with the velocities given by equations (\ref{eq_sphere_viscous_vel1}) and (\ref{eq_sphere_viscous_vel2}) and the pressure by equation (\ref{eq_sphere_slow_p}), the stress tensor components become
\begin{align*}
T_{rr}(r, \theta) & = -p_\infty + \frac{9U \mu a}{2r^2} \cos\theta - \frac{6 U \mu a^3}{2r^4} \cos \theta, \\
T_{\theta r}(r, \theta) & = -\frac{3 U \mu a^3}{2r^4} \sin\theta,
\end{align*}
and, when evaluated at $r = a$, we have
\begin{align}
\label{eq_stress_rr}
T_{rr}(a, \theta) & = -p_\infty + \frac{3 U \mu}{2a} \cos \theta, \\
\label{eq_stress_thetar}
T_{\theta r}(a, \theta) & = -\frac{3 U \mu }{2a} \sin\theta.
\end{align}


\subsubsection{The Force on the Sphere}

We're now finally in a position to calculate the force on the surface of the sphere from the surrounding fluid.  In terms of the stress tensor, this force is given by equation (\ref{eq_stress_tensor_force}),
\[
\vec{F} = \int_S (\tens{T} \cdot \unit{n} ) \, da.
\]
The surface of the sphere has a normal $\unit{n} = \unit{r}$, and remember that the dot product like this between a tensor and a vector gives us another vector, not a scalar.  This vector is the \emph{stress tensor} (see Section \ref{sec_stress}), which here I'll call $\vec{t_r}$:
\[
\vec{t_r} = \tens{T} \cdot \unit{r}.
\]
For clarity, I'll write out this calculation in matrix format so we can see exactly what is going on.  The stress vector then has components
\[
\vec{t_r} = \begin{bmatrix}
T_{rr} & T_{r\theta} & T_{r\phi} \\
T_{\theta r} & T_{\theta \theta} & T_{\theta \phi} \\
T_{\phi r} & T_{\phi \theta} & T_{\phi \phi} 
\end{bmatrix}
\begin{bmatrix}
1 \\
0 \\
0
\end{bmatrix} = \begin{bmatrix}
T_{rr} \\
T_{\theta r} \\
T_{\phi r}.
\end{bmatrix}
\]
In other words, the dot product picks out the components of the stress tensor that apply to a surface in the radial $\unit{r}$ direction.  The net force is then the surface integral over each component.

But we know from symmetry arguments that the net force on the sphere should be in the $z$ direction, so we can shorten our work a bit by only considering the $z$ component of this stress vector, in which case we want to calculate
\[
F_z = \int_S (\vec{t_r} \cdot \unit{z} ) \, da.
\]
In spherical coordinates, $\unit{z} = \cos\theta \unit{r} - \sin\theta \unit{\theta}$, so doing the dot product gives us
\[
F_z = \int_S ( T_{rr} \cos \theta - T_{\theta r} \sin \theta ) \, da.
\]
As mentioned above, we only need those two components of the stress tensor, coming from the direction of the surface ($\unit{r}$) and the direction of the net force (in both the $\unit{r}$ and $\unit{\theta}$ directions).

Plugging in the components $T_{rr}$ and $T_{\theta r}$ (evaluated at $r=a$) and using $da = a^2 \sin\theta \, d\theta d\phi$ for the surface element of the sphere gives us the integral
\[
F_z = \int_0^{2\pi} \int_0^\pi \left( -p_\infty \cos\theta + \frac{3U \mu }{2a} \cos^2\theta + \frac{3U\mu }{2a} \sin^2\theta \right) a^2 \sin\theta \, d\theta d\phi.
\]
This isn't a difficult integral to do; the term with $p_\infty$ integrates to zero, and the rest combine to give
\begin{equation}
\boxed{
F_z = 6 \pi U \mu a.
}
\end{equation}
This is our big result -- it gives the drag force on a sphere moving through a very viscous fluid, and is called \emph{Stokes law}.  Notice that the force is proportional to the velocity $U$ as you might expect.



%
% --- SECTION - ADVANCED TOPIC 5 ---
% 

\section{Stellar Structure}

In this section we turn our attention to using fluid dynamics to explore the structure of \emph{stars}.  There are two pretty big things which we'll have to include in addition to the usual theory:  the presence of \emph{self-gravity}, meaning that the star is massive enough that we have to take its own gravity into account, and the presence of energy generation and transport -- stars create energy through nuclear fusion at their core and transport it to the surface via mechanisms like radiative transport and convection.

\subsection{Newtonian Gravity}

We've already discussed, way back in Chapter 3, using a \emph{gravitational potential} to describe gravity, and we'll continue that here.  If $\vec{g}$ is the gravitation field as usual, then we can write it in terms of a gradient of a scalar,
\begin{equation}
\vec{g} = -\grad \Phi.
\end{equation}
It'll be much easier to work with the gravitational potential rather than the field.  

One difference from what we've done before, however, is the form that the field $\vec{g}$ takes; after all, we're not on the surface of the Earth anymore, where $\vec{g} = [0, 0, -g]$, for example.  Instead, it's due to the entire star itself, and we have to add up all the contributions from each bit of it.  From Newton's law of gravity, the force on a mass $m$ due to a mass $m'$ is 
\[
\vec{F}(\vec{r}) = -\frac{G m m'}{|\vec{r} - \vec{r}'|^3} (\vec{r} - \vec{r}'),
\]
where $\vec{r}$ is the location of mass $m$ and $\vec{r}'$ is the location of mass $m'$.  From Newton's second law, the mass feels an acceleration
\[
\vec{g}(\vec{r}) = \frac{\vec{F}(\vec{r})}{m} = -\frac{G m'}{|\vec{r} - \vec{r}'|^3} (\vec{r} - \vec{r}')
\]
(and $\vec{g}$ is often called the gravitational field).  At this point it's simple to generalize to get the field due to a continuous mass distribution like a star:
\begin{equation}
\vec{g}(\vec{r}) = -G \int_V \frac{ \rho(\vec{r}') (\vec{r} - \vec{r}')}{|\vec{r} - \vec{r}'|^3} \, d\tau'.
\end{equation}

This is the connection between the mass density $\rho(\vec{r})$ and the field $\vec{g}(\vec{r})$; what about the potential $\Phi(\vec{r})$?  We find the potential via Poisson's equation,
\begin{equation}
\label{eq_poisson}
\boxed{
\nabla^2 \Phi(\vec{r}) = 4 \pi G \rho(\vec{r}).
}
\end{equation}
This is the main equation of Newtonian gravity.


\subsection{Energy Transport}

A star generates energy at its core through fusion; stars on the so-called main sequence burn hydrogen to helium.  This process is of course quite complicated, requiring understanding of the various nuclear processes that happen, but if we ignore the details, suppose that $\epsilon$ is the total energy released through fusion per unit mass per unit time -- that means
\[
\int_V \epsilon \rho \, d\tau
\]
is the \emph{rate} at which energy is generated within the star.

The star then transports that energy to the surface.  There are two main processes that accomplish this:  radiative transport, in which energy is carried by photons, and adiabatic convection, in which energy is carried by matter.  Again, ignoring the details for now, we'll call $\vec{F}$ the \emph{flux} of energy, and the quantity
\[
\int_S \vec{F} \cdot d\vec{a}
\]
represents the rate that energy is transported through a surface $S$ with area vector $d\vec{a}$.

In addition to these two processes, the energy within a star can change due to the work done by the pressure in the fluid and the work done by gravity.  If we add up all the contributions, the total rate of energy change is
\begin{equation}
\label{eq_energy_all}
\frac{dE}{dt} = - \int_S p \vec{u} \cdot d\vec{a} + \int_V \vec{g} \cdot \vec{u} \rho d\tau + \int_V \epsilon \rho d\tau - \int_S \vec{F} \cdot d\vec{a}
\end{equation}
(note the negative sign on the pressure and flux terms to account for the fact that the area vector $d\vec{a}$ points outward).

The total energy $E$ can be written as the sum of the kinetic energy $T$ and the internal thermal energy (per unit mass) $U$, so the change in energy is
\[
\frac{dE}{dt} = \frac{d}{dt}\int_V \left( \frac{1}{2} \vec{u}^2 + U \right) \rho d\tau.
\]
If we move the derivative inside the integral, it becomes the material derivative, 
\[
\frac{dE}{dt} = \int_V \left[ \frac{D}{Dt}\left( \frac{1}{2} \vec{u}^2 \right) + \frac{DU}{Dt} \right] \rho d\tau.
\]
We can then use the divergence theorem to turn the surface integrals in equation (\ref{eq_energy_all}) into volume integrals, and, since the volume is arbitrary, drop the integrals to get
\begin{equation}
\label{eq_energy_all2}
\rho \left[ \frac{D}{Dt}\left( \frac{1}{2} \vec{u}^2 \right) + \frac{DU}{Dt} \right] = -\grad \cdot (p\vec{u}) + \rho \vec{g} \cdot \vec{u} + \rho \epsilon - \grad \cdot \vec{F}.
\end{equation}

One last thing:  we can use Euler's equation (assuming the star is made of an ideal fluid, a good assumption) to get rid of the kinetic energy term, and the continuity equation to clean things up a bit more.  Problem \ref{prob_energy_change} has the details; equation (\ref{eq_energy_all2}) becomes
\begin{equation}
\label{eq_thermal_energy}
\boxed{
\frac{DU}{Dt} = \frac{p}{\rho^2} \frac{D\rho}{Dt} + \epsilon - \frac{1}{\rho} \grad \cdot \vec{F}.
}
\end{equation}
This equation says that the internal thermal energy of a fluid element can change in three different ways:  by a change in density, by generating energy via fusion, or by transporting energy away. 

\subsection{Equations of Stellar Structure}

Equations (\ref{eq_poisson}) and (\ref{eq_thermal_energy}), along with Euler's equation and the continuity equation, constitute the theory of stellar structure.  For convenience, I'll put them all together:
\begin{align*}
\frac{D \uu}{Dt} = -\frac{1}{\rho} \grad p - \grad \Phi & \quad \text{(Euler's equation),} \\
\frac{\partial \rho}{\partial t} + \vec{\nabla} \cdot (\rho \vec{u} ) = 0 & \quad \text{(continuity equation),} \\
\nabla^2 \Phi = 4 \pi G \rho & \quad \text{(self-gravity),} \\
\frac{DU}{Dt} = \frac{p}{\rho^2} \frac{D\rho}{Dt} + \epsilon - \frac{1}{\rho}  \grad \cdot \vec{F} & \quad \text{(thermal enegry).}
\end{align*}

Now, that's four equations, but if you count the unknowns, there are actually five:  the velocity $\vec{u}$, the pressure $p$, the density $\rho$, the gravitational potential $\Phi$, and the thermal energy $U$.  Note that $\epsilon$, the rate of energy generation, and $\vec{F}$, the flux of energy, aren't strictly unknowns here; they require additional theory to compute, depending on what exactly we're modelling, so think of them as \emph{inputs} into the theory of stellar structure. But what's the fifth equation?  As discussed back in Section \ref{sec_comp_fluids}, we need an \emph{equation of state}.  More on what we'll use in a bit.

These equations are extremely complicated to solve, even numerically.  But we can make some headway and produce simple but useful models of stars and other objects with a few assumptions:
\begin{itemize}
\item That the fluid has no time dependence -- it's \emph{steady flow}.  But we'll also assume no time dependence for every variable, including energy $U$.
\item Even stronger, we'll assume the fluid is \emph{static} -- that $\vec{u} = 0$.  Our stars will be made up of a fluid that doesn't actually flow.
\item And we'll assume spherical symmetry -- there's only a dependence on the distance from the origin, $r$.
\end{itemize}

As you might guess, these assumptions drastically simplify our equations; they become
\begin{align}
\frac{dp}{dr} = -\rho \frac{d\Phi}{dr}  & \quad \text{(Euler's equation),} \label{eq_star_1} \\
0 = 0 & \quad \text{(continuity equation),} \\
\frac{1}{r^2} \frac{d}{dr} \left(r^2 \frac{d\Phi}{dr} \right) = 4 \pi G \rho & \quad \text{(self-gravity),} \label{eq_star_3} \\
\epsilon = \frac{1}{\rho} \frac{1}{r^2} \frac{d}{dr} ( r^2 F) & \quad \text{(thermal enegry).} \label{eq_star_4}
\end{align}
But even better, we can combine the first and third equation in the following way.  First, integrate the third equation for self-gravity, so that
\[
r^2 \frac{d\Phi}{dr} = \int_0^r 4\pi G \rho(r') r'^2 \, dr'.
\]
But the right hand side is (almost) just the mass contained inside a radius $r$:
\[
M(r) = 4\pi \int_0^r \rho(r') r'^2 \, dr',
\]
so the third equation can be written
\begin{equation}
\frac{d\Phi}{dr} = \frac{GM(r)}{r^2},
\end{equation}
and substituted into the first equation, giving
\begin{equation}
\label{eq_star_1b}
\boxed{
\frac{dp}{dr} = - \frac{GM(r) \rho}{r^2}.
}
\end{equation}
This is the equation for \emph{hydrostatic equilibrium}: the inward force of gravity is balanced by the outward pressure of the fluid, allowing the star to be static.

The fourth equation, for thermal energy, says that any energy produced ($\epsilon$) within the star must be balanced by the energy flux ($F$) -- this is a statement of \emph{thermal equilibrium}.  It's more common to write this in terms of the \emph{luminosity} $L$, which is the energy emitted per unit time by a radial shell of the fluid, in which case the flux is
\begin{equation}
F = \frac{L}{4\pi r^2}.
\end{equation}
Then the fourth equation becomes, in terms of the luminosity,
\begin{equation}
\boxed{
\frac{dL}{dr} = 4\pi r^2 \rho \epsilon.
}
\end{equation}

Finally, we need to discuss the equation of state -- but there are two possibilities.  If we assume the pressure in the fluid is provided purely by the radiation emitted by the gas, then the equation of state is
\begin{equation}
p_\text{rad} = \frac{1}{3} aT^4,
\end{equation}
where $a = 7.566 \times 10^{-16}$ J m$^{-3}$ K$^{-4}$ is called the radiation constant (it's related to the Stefan-Boltzmann constant if you're more familiar with that).  There could also be a contribution to the pressure from the fluid itself.  If we assume it's an ideal gas, then we have the usual (see Section \ref{sec_comp_fluids}) gas law
\begin{equation}
p_\text{gas} = \rho \kappa T.
\end{equation}

In general, both of these sources of pressure are important, and we should include them both:
\begin{equation}
p = p_\text{rad} + p_\text{gas} = \frac{1}{3} aT^4 + \rho \kappa T.
\end{equation}
Suppose that the gas pressure makes up some fraction of the total -- call the fraction $\beta$, so that
\[
p_\text{gas} = \beta p = \rho \kappa T.
\]
Radiation pressure makes up the rest:
\[
p_\text{rad} = (1 - \beta) p = \frac{1}{3} aT^4.
\]
We can use these two equations to eliminate the temperature, giving the pressure in terms of the density alone.  It turns out to take on the simple form
\begin{equation}
\label{eq_eos_edd}
p = k \rho^{4/3},
\end{equation}
where the constant $k$ is given by
\[
k = \left[ \frac{3 (1-\beta) \kappa^4}{a \beta^4} \right]^{1/3}.
\]
Stars with an equation of state given by equation (\ref{eq_eos_edd}) are called \emph{Eddington standard models}.  But it turns out that the general form of the equation of state, that $p \propto \rho^\gamma$, arises in many situations -- we've seen one already in Section \ref{sec_comp_fluids} with adiabatic processes, where $\gamma$ is related to the ratio of specific heats.  Models which have this form are called \emph{polytropes} and deserve a bit more discussion.

\subsection{Polytropic Models}

Suppose, then, we take as our equation of state the general form
\begin{equation}
\label{eq_polytrope}
p = k \rho^\gamma.
\end{equation}
What does this do for us?  We can use it to eliminate either pressure or density from equation (\ref{eq_star_1b}).  First rearrange and take the derivative of equation (\ref{eq_star_1b}) to get
\[
\frac{d}{dr} \left( \frac{r^2}{\rho} \frac{dp}{dr} \right) = -G \frac{dM}{dr},
\]
and then use $dM/dr = 4\pi r^2 \rho$ to get
\[
\frac{d}{dr} \left( \frac{r^2}{\rho} \frac{dp}{dr} \right) = - 4 \pi G \rho.
\]
Now we can replace the pressure using equation (\ref{eq_polytrope}); the derivative of it is
\[
\frac{dp}{dr} = k \gamma \rho^{\gamma - 1} \frac{d\rho}{dr}.
\]
Plugging this into the equation above gives us an expression involving only the density:
\begin{equation}
\label{eq_star_density}
\frac{k\gamma}{r^2} \frac{d}{dr} \left( r^2 \rho^{\gamma - 2} \frac{d\rho}{dr} \right) = -4\pi G \rho.
\end{equation}

We could in principle solve this differential equation for the density $\rho(r)$, but this is not a very simple-looking equation.  We can clean it up a bit by introducing two dimensionless parameters.  The first is a scaled density parameter, $\varrho$, such that
\begin{equation}
\label{eq_varrho}
\rho(r) = \rho_c \varrho^n(r),
\end{equation}
where $\rho_c$ is a constant (with units of density) and the index $n$ is related to the parameter $\gamma$ by
\begin{equation}
\gamma = \frac{n+1}{n}.
\end{equation}
The second parameter is a scaled radius; define 
\begin{equation}
\eta = \frac{r}{\lambda},
\end{equation}
where $\lambda$ is a constant dependent on the index $n$:
\begin{equation}
\lambda^2 = \frac{(n+1) k \rho_c^{(1-n)/n}}{4\pi G}.
\end{equation}
In terms of these two new variables, equation (\ref{eq_star_density}) becomes
\begin{equation}
\boxed{
\frac{1}{\eta^2} \frac{d}{d\eta} \left( \eta^2 \frac{d\varrho}{d\eta} \right) = - \varrho^n.
}
\end{equation}
This is called the \emph{Lane-Emden equation}, and the solutions to it are polytropic models of stars.

We'll solve the Lane-Emden equation in a moment, but first we need to discuss the boundary conditions.  The first is pretty simple: we want the density at the centre of the star to be \emph{finite}, so we'll choose
\begin{equation}
\varrho(0) = 1.
\end{equation}
This makes the constant $\rho_c$ the value of the density at the centre of the star (hence the subscript $c$), something that can be fit to model a particular system.  The second boundary condition needs a bit more motivation.  

Consider a small volume at the centre of the star with radius $\delta$.  The volume within $\delta$ is
\[
V = \frac{4}{3} \pi \delta^3,
\]
and the mass enclosed by that radius is
\[
M = \frac{4\pi}{3} \bar{\rho} \delta^3,
\]
where $\bar{\rho}$ is the \emph{average} density within the volume.  From equation (\ref{eq_star_1b}), the change in pressure is then
\[
\frac{dp}{dr} = -\frac{4\pi}{3} G \bar{\rho}^2 \delta.
\]
In the limit as $\delta \to 0$, then, we have that
\[
\frac{dp}{dr} \to 0.
\]
Now, for a polytropic model, the density is proportional to the pressure, so this also says
\[
\frac{d\rho}{dr} \to 0 \quad \text{as} \quad r \to 0.
\]
In terms of the dimensionless variables, our second boundary condition becomes
\begin{equation}
\frac{d\varrho}{d\eta} = 0 \quad \text{at} \quad \eta = 0.
\end{equation}

We're ready to solve the Lane-Emden equation and produce our first models for a star.  It turns out, however, that there's only a few cases for which it can be solved analytically:  for values of $n = 0, 1, 5,$ and $n = \infty$.  I'll walk you through a couple here, and you can work on Problem \ref{prob_lane_emden} for others.

\begin{example}[n = 0]
The easiest case is $n = 0$, in which case the Lane-Emden equation becomes
\[
\frac{1}{\eta^2} \frac{d}{d\eta} \left( \eta^2 \frac{d\varrho}{d\eta} \right) = -1.
\]
This case be solved by integrating twice, giving
\[
\varrho(\eta) = c_1 - \frac{c_2}{\eta} - \frac{1}{6} \eta^2.
\]
Fitting the boundary conditions requires $c_1 = 1$ and $c_2 = 0$, and the dimensionless density is
\[
\varrho(\eta) = 1 - \frac{1}{6} \eta^2.
\]
Notice that the density drops to zero at a radius $\eta_s = \sqrt{6}$ -- this is the surface of the star.  

But this is a strange model for star, since, according to equation (\ref{eq_varrho}), the density is actually constant $\rho_c$ throughout.  So, not a very good model of a star -- but it does turn out to be a reasonable model of a rocky planet like the Earth.
\end{example}

\begin{example}[n = 5]
If $n = 5$, the Lane-Emden equation becomes
\[
\frac{1}{\eta^2} \frac{d}{d\eta} \left( \eta^2 \frac{d\varrho}{d\eta} \right) = -\varrho^5.
\]
This has the solution
\[
\varrho(\eta) = \frac{1}{\sqrt{1 + \eta^2/3}}.
\]
Unlike the previous example, where the density dropped to zero at a certain point, in this case the density becomes zero at infinity -- evidently $n=5$ models an infinitely large star!  Actually, this case -- usually called a \emph{Plummer sphere} -- can be a simple model of systems of stars like globular clusters or galaxies.
\end{example}

Neither of these two examples were particularly good model for \emph{stars}, the objects we set out to investigate in the first place.  Other values of $n$ do a better job of that:  $n = 1.5$ provides a good model of convective star cores as well as low mass white dwarfs, for example, and $n = 3$, which corresponds to the Eddington standard model, models the radiation zones of main-sequence stars as well as high mass white dwarfs.  However, those cases don't have an analytic solution and the Lane-Emden equation must be solved numerically.  The densities of some of these models are shown in Figure \ref{fig_polytropes}.

\begin{figure}
\centering
\includegraphics[width=0.9\linewidth]{Figures/Chapter6/fig_polytropes}
\caption{Density profiles for various polytropes.  Left: the dimensionless density parameter $\varrho(\eta)$.  Right: the density $\rho(r)$, using solar values for various constants, plotted in terms of the central density of the sun and solar radii.  Note that the horizontal axis is log scaled. Blue: $n = 1$.  Orange: $n = 1.5$. Green: $n=3$.  Red: $n=5$. }
\label{fig_polytropes}
\end{figure}


%
% --- SECTION - ADVANCED TOPIC 6 ---
% 

\section{Computational Methods}

Although we've now spent almost 200 pages talking about and solving fluid dynamics problems, it should be clear that what we've covered is actually drastically limited in scope.  Here's what we've been able to solve:
\begin{itemize}
\item Viscous fluid flowing in a straight line or in a circle
\item Ideal flow past various simple objects, like circular cylinders or infinitely long aerofoils
\item The motion and interaction of idealized vortexes
\item Small amplitude surface waves
\item Small amplitude sound waves
\item The free surface of a rotating fluid or a dam break
\item And a few other advanced topics in this chapter: a simple boundary layer model, very viscous flow past a sphere, a simple model for a static star.
\end{itemize}

That list is both impressive and also short.  Although there's more theory we could talk about -- boundary layers could fill its own chapter, and we could at least touch on instability and turbulence -- the fact is that most applications of fluid dynamics today are much too complicated to handle by solving equations by hand, even with the impressive set of tools we've learned here.  If we want to understand the lift and drag forces on a real airplane, model weather and climate, examine high-speed plasma outflows from accretion disks surrounding black holes, or check how close our dam breaking theory matches an actual dam break (without causing one in actuality), we need to turn to computational methods of solving the Navier-Stokes and related equations. 

These methods are outside the scope of this textbook, so I won't be providing too much detail, but they're worth discussing even a little because of their importance.  

\subsection{Computational Fluid Dynamics}

Traditionally, computational methods involved solving the Navier-Stokes equation, along with the continuity equation and conservation of energy, by discretizing them on a grid of some sort.  This process can be made computationally easier by combining all three of these equations into one:
\begin{equation}
\label{eq_ns_comp}
\dfdx{(\rho \phi)}{t} + \grad \cdot (\rho \phi \vec{u}) = \grad \cdot (\Gamma \grad \phi) + S_\phi,
\end{equation}
where the variable\footnote{This $\phi$ is not the spherical coordinate or the velocity potential, sorry once again for overusing this letter.} $\phi$ stands for either $1$ (in which case this equation reduces to the continuity equation), a component of velocity (for example, if $\phi = u$, the equation reduces to the $x$ component of the Navier-Stokes equation), or the internal energy (in which case it becomes conservation of energy).  The variable $\Gamma$, called the diffusion coefficient, is just the viscosity if $\phi$ is a velocity component and is the thermal conductivity if it is the energy.  The last term, $S_\phi$, is called the \emph{source term}, and could represent a force or an energy source.

Turning this equation into a form suitable for solving it computationally is a complex problem, and various schemes are in use.  The simplest is the \emph{finite difference method}, in which quantities are evaluated on a grid in space.  The derivatives in equation (\ref{eq_ns_comp}) become finite differences (hence the name); for example, the \emph{forward difference} scheme approximates
\[
\frac{df}{dx} \approx \frac{f(x+\Delta x) - f(x)}{\Delta x}.
\]
Any ordinary differential equations I solved numerically in the course of this textbook were done using a finite difference method.  For the Navier-Stokes equation, though, this method is not particularly flexible and isn't often used except in cases where a regular grid is appropriate -- for example, simulations of weather.

The \emph{finite volume method} is more popular for solving fluid dynamics problems.  In this method, the fluid domain is divided into small \emph{control volumes}; for each control volume -- or \emph{cell} -- the volume average of the fluid variables can be evaluated.  Variables can change due to a flux entering into the volume, and this must be balanced by a corresponding flux out of the volume to ensure conservation of that quantity.  This method requires rewriting equation (\ref{eq_ns_comp}) by integrating over the volume, and using the divergence theorem to reduce the volume integrals to surface integrals where there are divergences.  The main equation then becomes
\[
\dfdx{}{t} \left( \int_V \rho \phi \, d\tau \right) =- \int_S \rho \phi \vec{u} \cdot d\vec{a} + \int_S \Gamma \grad \phi \cdot d\vec{a} + \int_V S_\phi \, d\tau,
\]
which we can interpret as saying that any increase in $\phi$ inside the cell must be because of \emph{convection} (first term on the right hand side) or \emph{diffusion} (second term) or the generation of more $\phi$ (the third term).  Finite volume methods are more flexible than finite differences, and can be used with both a regular mesh of cells or an irregular one -- useful in solving problems involving complicated shapes like the surface of a car. 

A third technique is the \emph{finite element method}, in which the fluid domain is again divided up into smaller elements.   However, unlike the previous two methods, each element is modelled independently assuming each variable changes according to a simple function.  The models for each element are then combined into a larger system of equations that models the entire system.

If you're interested in exploring computational fluid dynamics, I recommend the software SU2 (\href{https://su2code.github.io}{su2code.github.io}), an open source package written in c++ and Python.  SU2 can solve Euler's equation and the Navier-Stokes equation for both incompressible and compressible fluids, and is very flexible:  although designed originally for aerodynamic shape optimization, it can handle potential flow, elasticity, and flows containing electrodynamics or chemical reactions.  Figure \ref{fig_vortex_shedding} was created using SU2 -- it solved the Navier-Stokes equations on an irregular multi-level grid using a finite volume method for flow past a cylinder.

\subsection{Smoothed Particle Hydrodynamics}

A very different approach to solving the Navier-Stokes and related equations is taken by \emph{smoothed particle hydrodynamics,} first developed to simulate polytropic models of stars. Rather than using a grid or mesh, this method uses particle-based techniques to simulate interactions between volume elements -- sometimes in this case called \emph{control masses.}  The particles carry all physical properties of the system as they move; this viewpoint, in which the coordinate system follows the particles, is called \emph{Lagrangian}.  Throughout this textbook, on the other hand, I've used a \emph{Eulerian} viewpoint in which the fluid properties are described at fixed points.

How exactly does this work?  Well, consider for simplicity Euler's equation,
\[
\frac{D\vec{u}}{Dt} = -\frac{1}{\rho} \grad p + \frac{1}{\rho} \vec{f},
\]
where $\vec{f}$ represents any external forces present (for example, gravity would be $\rho \vec{g}$ as usual).  Because of the Lagrangian viewpoint, we can treat the material derivative as $d\vec{u}/dt$, where $\vec{u} = d\vec{r}/dt$.  Given an initial configuration of the fluid, represented as particles at particular positions, Euler's equation tells us how to update those positions and other quantities by approximating the derivatives as finite differences.  One particular particle, for example, would have its position and velocity updated according to
\begin{align*}
\Delta \vec{r} &= \vec{u} \Delta t, \\
\Delta \vec{u} &= \left( -\frac{1}{\rho} \grad p + \frac{1}{\rho} \vec{f} \right) \Delta t,
\end{align*}
where $\Delta t$ is the time step.

One issue with this scheme, however, is how to compute the derivatives that shown up in Euler's equation (or the Navier-Stokes, which has a second order derivative).  Consider the pressure; it can be written in terms of a \emph{interpolating or smoothing kernel} $W(\vec{r})$ as
\[
p(\vec{r}) \approx \int p(\vec{r'}) W(\vec{r} - \vec{r'}) \, d\tau'.
\]
Note that if $W(\vec{r}) = \delta^3(\vec{r})$, then this relation is exact; most commonly the kernel is a Gaussian,
\[
W(\vec{r}) = \left( \frac{1}{h \sqrt{\pi} } \right)^3 e^{-r^2/h^2},
\]
where $h$ is the smoothing length scale (in the limit $h\to 0$, the Gaussian becomes the Dirac delta).  Now, if we want to calculate the gradient of the pressure, the $\grad$ operator only acts on function of the coordinate $\vec{r}$ -- so we can write
\[
\grad p(\vec{r}) = \int p(\vec{r'}) \grad W(\vec{r} - \vec{r'}) \, d\tau,
\]
and perform the gradient by hand.  

As with traditional CFD methods, SPH must be discretized for calculation on a computer.  In this case it's straightforward -- the integrals become sums in the usual way.  The volume element $d\tau$ can be written in terms of the density,
\[
d\tau = \frac{ m}{ \rho},
\]
where $m$ is the mass of the particle under consideration, and the pressure gradient takes the form
\[
\grad p \approx \sum_j m \frac{p(\vec{r}_j)}{\rho_j} \grad W(\vec{r} - \vec{r}_j).
\]
The sum is over all particles, and $\rho_j$ is the density of the $j$th particle, which can be found from
\[
\rho_i = \sum_j m_j W(\vec{r}_i - \vec{r_j}).
\]
Finally, then, Euler's equation for updating the velocity of the $i$th particle becomes
\begin{equation}
\frac{\Delta \vec{u}_i}{\Delta t} = -\frac{1}{\rho_i} \sum_j m_j \frac{p_j}{\rho_j} \grad W(\vec{r}_i - \vec{r}_j) + \frac{1}{\rho_i} \vec{f}_i
\end{equation}
(actually, this is usually written in a way that's more efficient for calculation, but this should give you an idea of what's involved).  If viscous forces are to be included, the Laplacian in the Navier-Stokes equation is handled in a similar manner to the gradient here.

If you're interested in exploring smoothed particle hydrodynamics, I recommend taking a look at PySPH (\href{https://pysph.readthedocs.io}{pysph.readthedocs.io}), an open source SPH package written in Python.  It can handle weakly compressible SPH for free surface flows (like the dam break shown in Figure \ref{fig_sph_dam}), incompressible fluids, elastic dynamics, various boundary conditions, and more.  


\begin{figure}
\centering
\includegraphics[width=\linewidth]{Figures/Chapter6/fig_sph_dam.png}
\caption{Smoothed particle hydrodynamics simulation of the two dimensional dam break problem using PySPH.  The red colour indicates faster horizontal speed $u$. }
\label{fig_sph_dam}
\end{figure}


\subsection{Lattice Boltzmann Method}

Now for something completely different -- the lattice Boltzmann method doesn't even try to solve the Navier-Stokes equation.  It treats the fluid on a grid -- called a lattice in this case -- but limits fluid particles to only two behaviours: they can \emph{stream} or they can \emph{collide}.  It does this by considering the collisional Boltzmann equation,
\begin{equation}
\label{eq_boltzmann}
\dfdx{f}{t} + (\vec{u} \cdot \grad) f + \grad \Phi \cdot \dfdx{f}{\vec{u}} = \left( \dfdx{f}{t} \right)_\text{coll},
\end{equation}
which is an equation for the \emph{distribution function} $f(\vec{r}, \vec{u}, t)$.  This function describes the phase-space density of the fluid.  This means that the quantity 
\[
f(\vec{r}, \vec{u}, t) \, d^3\vec{r} \, d^3\vec{u}
\]
is the mass of fluid particles having positions in the volume $d^3\vec{r}$ centred on $\vec{r}$ and with velocities in the range $d^3\vec{u}$ centred on $\vec{u}$.  Any physical quantity can be calculated from the distribution function; for example, the mass density is 
\[
\rho(\vec{r}, t) = \int f(\vec{r}, \vec{u}, t) \, d^3\vec{u}.
\]

In equation (\ref{eq_boltzmann}), you might recognize the material derivative of $f$ on the left hand side.  In the third term, $\Phi$ represents the potential due to some external force, if one is acting.  And the term on the right hand side is there to handle interactions between fluid particles -- in general, this term can be very difficult to calculate.  The so-called BGK approximation (from Bhatnagar, Gross, and Krook) models the collisions as
\[
\left( \dfdx{f}{t} \right)_\text{coll} = -\frac{f - f_\text{eq}}{\tau},
\]
where $f_\text{eq}$ is an equilibrium state and $\tau$ is the timescale for collisions to occur.  A common equilibrium state is an isothermal one, so that the collisions try to drive the fluid towards a state where the temperature is the same everywhere.

If we drop the external potential term and use the BGK model, the Boltzmann equation becomes
\[
\dfdx{f}{t} + (\vec{u} \cdot \grad) f = -\frac{f - f_\text{eq}}{\tau}.
\]
The left hand side describes the movement of fluid particles and the right their collisions.  This is the heart of the lattice Boltzmann method.

As with other methods, though, this equation needs to be discretized on a lattice so that the distribution function $f$ can be evolved in time.  Fluid quantities can then be found by summing over the nearby neighbours of the lattice site; for example, the density is
\[
\rho = \sum_i f_i,
\]
and the velocity is
\[
\vec{u} = \frac{1}{\rho} \sum_i f_i \vec{u}_i.
\]

This method can be simple to implement and, in cases where it's applicable, much faster than standard computational methods.  There are various software packages available, but a simple open source implementation is given by Philip Mocz (\href{https://github.com/pmocz/latticeboltzmann-python}{github.com/pmocz}).  Figure \ref{fig_lbm} shows a simulation based on this code for fluid flowing past a cylinder; streamlines and vorticity are shown for different times.  You can compare with Figure \ref{fig_vortex_shedding}, which does the same thing but with standard CFD.  The lattice Boltzmann method does a good job with vortex shedding and the formation of a vortex street.

\begin{figure}
\centering
\includegraphics[width=0.8\linewidth]{Figures/Chapter6/fig_lbm}
\caption{Lattice Boltzmann method simulation of fluid flowing past a cylinder, shown at four different times.  Streamlines are shown, in addition to the vorticity in the fluid (with red being positive vorticity and blue negative).}
\label{fig_lbm}
\end{figure}

\section*{Problems}
\addcontentsline{toc}{section}{Problems}
\markright{Problems}%

\begin{problem}[Pressure and the Stress Tensor]
\label{prob_pressure} 
Show that, for the Newtonian stress tensor (equation \ref{eq_stress_newtonian}), the pressure is the negative of the average of the three normal stresses.  Explain why this makes sense.
\end{problem}

\begin{problem}[Flow around a sphere]
\label{prob_sphere} 
Consider ideal incompressible flow, uniform at infinity, around a sphere of radius $a$, as in Section \ref{sec_sphere_ideal}.

(a) From the velocity potential $\varphi(r, \theta)$ in equation \ref{eq_sphere_pot}, show that the velocity of the flow is
\[
u_r (r, \theta) = U \cos \theta \left( 1 - \frac{a^3}{r^3} \right) \quad \text{and} \quad u_\theta(r, \theta) = -U \sin \theta \left( 1 + \frac{a^3}{2r^3} \right).
\]

(b) Show in spherical coordinates that the Stokes stream function satisfies
\[
(\vec{u} \cdot \grad) \Psi = 0
\]
and is thus constant on streamlines.

(c) Show that for the flow around a sphere, the Stokes stream function can be written as
\[
\Psi (r, \theta) = -\frac{1}{2} U \left( r^2 - \frac{a^3}{r} \right) \cos^2 \theta
\]
or
\[
\Psi (r, \theta) = \frac{1}{2} U \left( r^2 - \frac{a^3}{r} \right) \sin^2 \theta.
\]
Are these equivalent?  Why or why not?
\end{problem}



\begin{problem}[Terminal velocity for a dropped sphere]
\label{prob_terminal_vel} 
Suppose you drop a steel sphere, of radius $1$ cm, into a bucket of glycerine.  What is the terminal speed of the sphere in the glycerine in cm/s?
\end{problem}




\begin{problem}[Rate of change of thermal energy]
\label{prob_energy_change} 

(a) From Euler's equation (\ref{eq_euler}), derive the \emph{mechanical energy equation},
\[
\frac{D}{Dt} \left( \frac{1}{2} \vec{u}^2 \right) = -\frac{1}{\rho} \vec{u} \cdot \grad p + \vec{u} \cdot \vec{g}.
\]
(\emph{Hint}: Start by taking the dot product of Euler's equation with $\vec{u}$.)

(b) From the continuity equation (\ref{eq_continuity}), show that you can write the divergence of the velocity as
\[
\grad \cdot \vec{u} = -\frac{1}{\rho} \frac{D\rho}{Dt}.
\]

(c) Thus, show that equation (\ref{eq_energy_all2}) becomes equation (\ref{eq_thermal_energy}).

\end{problem}


\begin{problem}[Polytropes]
\label{prob_lane_emden} 
Consider a polytrope with $n=1$.  

(a) Solve the Lane-Emden equation to find the dimensionless density $\varrho(\eta)$ and calculate the value of the surface $\eta_s$.

(b) Show that the total mass within the star is $M = 4\pi^2 \rho_c \lambda^3$.

(c) Define the dimensionless pressure as 
\[
\mathscr{p} = \frac{p}{k\rho_c^n}
\]
and find $\mathscr{p}(\eta)$ for this case.

(d) Plot the dimensionless density and dimensionless pressure.
\end{problem}




